\documentclass{exam}
\usepackage[margin=1in]{geometry}
\usepackage{amsmath}
\usepackage{amssymb}
\usepackage{mdframed}

\newcommand{\R}{\mathbb{R}}
\newcommand{\N}{\mathbb{N}}
\newcommand{\Z}{\mathbb{Z}}
\newcommand{\Q}{\mathbb{Z}}
\newcommand{\?}{\stackrel{?}{=}}

\newcommand{\mybox}[1]{\fbox{\begin{minipage}#1\end{minipage}}}

\title{Applied Cryptography\\Homework 1}
%\author{Christian Mann}
\date{Due September 29, 2015}
\begin{document}
\maketitle
%\printanswers

% this is exposition for Question 1, but I can't put it in \begin{questions}

\begin{questions}

	\section*{Functions}

	\question 
	Let $f(x) = x^2 + x$ on $\R$. Is $f$ onto? Is $f$ one-to-one? Describe $f(P)$, where $P$ is the set of positive real numbers.
	\begin{solution}
		$f$ is not onto: $-5 \not\in \mathrm{Range}(f)$. \\
		$f$ is not one-to-one: $f(-1) = f(0) = 0$. \\
		$f(P)$ is described exactly by $P$.
	\end{solution}
	
	\begin{mdframed}
		We often are interested in the \textit{image} of a set under a function.
		The image of $P$ under $f$ is written as $f(P)$ and is equal to
		\[ f(P) = \{f(p) : p \in P\} \]
		
		We say that $A$ is a subset of $B$, or $A \subseteq B$, if every element in
		$A$ is in $B$. To prove this, assume $a \in A$ and prove that $a \in B$.

		$\alpha: S \rightarrow T$ means ``$\alpha$ is a function from $S$ to
		$T$.'' It
		does not claim anything regarding one-to-one, onto, or any other properties
		of the function.

	\end{mdframed}

	\question 
	Prove that if $\alpha: S \rightarrow T$ and $A$ and $B$ are subsets of $S$, then $\alpha(A \cup B) = \alpha(A) \cup \alpha(B)$.
	\begin{solution}
		Let $\alpha: S \rightarrow T$, $A \subseteq S$, $B \subseteq S$.

		Now let $y \in \alpha(A \cup B)$. So there exists a $y \in A \cup B$ s.t. $y = \alpha(x)$. Either $x \in A$ or $x \in B$. So either $\alpha(y) \in \alpha(A)$ or $\alpha(y) \in \alpha(B)$. Therefore $y \in \alpha(A) \cup \alpha(B)$.

		On the other hand, let $y \in \alpha(A) \cup \alpha(B)$. So either $y \in \alpha(A)$ or $y \in \alpha(B)$. So there exists an $x$ s.t. $y = \alpha(x)$ and either $x \in A$ or $x \in B$. So $x \in A \cup B$. Therefore $f(x) = y \in \alpha(A \cup B)$.
	\end{solution}

	\question 
	\begin{parts}
		\part Prove that if $\alpha: S \rightarrow T$, and $A$ and $B$ are subsets of $S$, then $\alpha(A \cap B) \subseteq \alpha(A) \cap \alpha(B)$.
		\begin{solution}
			Let $y \in \alpha(A \cap B)$. Then $\exists x \in A \cap B$ s.t. $y = \alpha(x)$. Then $x \in A$ and $x \in B$. So $y \in \alpha(A)$ and $y \in \alpha(B)$. Thus $y \in \alpha(A) \cap \alpha(B)$.
		\end{solution}
		\part Give an example (specific $S$, $T$, $A$, $B$, and $\alpha$) to show that equality need not hold in part (a).
		\begin{solution}
			Let $S = \{1,2\}$. Let $A = \{1\}$, $B = \{2\}$. Let $\alpha(1) = \alpha(2) = 1$. Then
			\begin{align*}
				A \cap B                 &= \varnothing \\
				\alpha(A \cap B)         &= \varnothing \\
				\alpha(A) \cap \alpha(B) &= \{1\} \cap \{1\} \\
				                         &= \{1\}
			\end{align*}
		\end{solution}
	\end{parts}

	\question 
	Which of the functions sine, cosine, and tangent, as mappings from $\R$ to $\R$, are invertible?
	\begin{solution}
		Sine and cosine are both not onto: $\lnot\exists x: \sin(x) = 2$ and $\lnot\exists x: \cos(x) = 2$. \\
		Tangent is not one-to-one: $\tan(0) = \tan(2\pi) = 0$. \\
		Therefore none are invertible.
	\end{solution}

	\section*{Operations}

	\question 
	Let $\star$ be defined as $m \star n = 2^{mn}$.
	\begin{parts}
		\part Is $\star$ an operation on the set of integers?
		\begin{solution}
			No:
			\begin{align*}
				m = -1&, n = 1 \\
			m \star n &= 2^{-1 \cdot 1} \\
					  &= 2^{-1} \\
					  &= \frac{1}{2} \not\in \Z
			\end{align*}
		\end{solution}
	\end{parts}

	\question 
	\begin{parts}
		\part Complete the Cayley table of $\star$ on $\{u, v\}$ in such a way that $u$ becomes an identity element. In how many ways can this be done?
			\[
			\begin{array}{c|cc}
				\hline
				\star & u & v \\
				\hline
				u &  &  \\
				v &  & 
			\end{array}
			\]
		\begin{solution}
			\[
			\begin{array}{c|cc}
				\hline
				\star & u & v \\
				\hline
				u & u & u \\
				v & u & \cdot
			\end{array}
			\]
			There are two ways to fill in the last entry.
		\end{solution}
		\part Can the table be completed in such a way that $u$ and $v$ both become identity elements?
		\begin{solution}
			No. It is not possible for two distinct elements to both be identity elements: if $u$ and $v$ are both identity elements, then $u \star v = u$ and $u \star v = v$; thus $u = v$.
		\end{solution}
		\part Prove: An operation $\star$ on a set $S$ (any $S$) can have at most one identity element.
		\begin{solution}
			See (b).
		\end{solution}
	\end{parts}

	\begin{mdframed}
		A set $S$ is \textit{closed} with respect to an operation $\star$ if for
		every $x, y \in S$, $x \star y \in S$.
	\end{mdframed}

	\question 
	Assume that $\star$ is an associative operation on $S$ and that $a \in S$. Let
		\[ C(a) = \{x: x \in S \,\mathrm{and}\, a \star x = x \star a\} \]
	Prove that $C(a)$ is closed with respect to $\star$.
	\begin{solution}
		First of all, we can see that $a \in C(a)$, since $a \star a = a \star a$.

		Let $b, c \in C(a)$. We know that $b \star c \in S$, since $\star$ is an operation on $S$.
		\begin{align*}
			b \star (a \star c) &= b \star (a \star c) \\
			(b \star a) \star c &= b \star (a \star c) \\
			(a \star b) \star c &= b \star (c \star a) \\
			a \star (b \star c) &= (b \star c) \star a \\
		\end{align*}
		Therefore $b \star c \in C(a)$.
	\end{solution}

	\section*{Groups}

	\begin{mdframed}
		A \textbf{group} is a set $G$ together with an operation $\star$ on $G$ such
		that each of the following three properties hold:
		
		\textit{Associativity}: $ a \star (b \star c) = (a \star b) \star c $
		
		\textit{Existence of an identity element}
			There is an element $e \in G$ such that $a \star e = e \star a = a$ for
			each $a \in G$.
		
		\textit{Existence of inverse elements}
			For each $a \in G$ there is an element $b \in G$ such that $a \star b =
			b \star a = e$.
		
		\textbf{Example}: The integers together with the addition operator form
		a group. You can see the three properties hold.

		\textbf{Example}: The set of even integers together with addition is a
		group. The sum of two even integers is an even integer. The associative law
		is true for all integers, so it is true for the subset of all even integers.
		The identity element is $0$ (an even integer), and the inverse of an even
		integer $x$ is $-x$ (again, an even integer).

		\textbf{Example}: The set of positive integers is not a group.

		\textbf{Example}: The set $\{0\}$ with addition is a group. Note: Because a
		group must contain an identity element, the set underlying a group must
		always contain at least one element.
	\end{mdframed}

	\question 
	Verify that $S = \{2^m3^n : m,n \in \Z\}$ is a group with respect to multiplication. Identify clearly the properties of $\Z$ and $\R$ that you use.
	\begin{solution}
		\begin{enumerate}
			\item Given any two elements, their product is in the group:
				\[ 2^a3^b \cdot 2^c3^d = 2^{a+c}3^{b+d} \]
				Therefore $S$ is closed under multiplication.
			\item The identity element is $1 = 2^03^0$. For any element $x \in S$, $1 \cdot x = x \cdot 1 = x$.
			\item For any element $2^m3^n$ in $S$, its inverse is in the group:
				\[ 2^m3^n \cdot 2^{-m}3^{-n} = 2^{m-m}3^{n-n} = 2^03^0 \]
			\item Multiplication is associative, as a property of $\R$.
		\end{enumerate}
	\end{solution}

	\question 
	Let $G$ denote the set of all $2 \times 2$ real matrices $A$ with $\det(A) \not= 0$ and $\det(A) \in \Q$ (the rational numbers).
	\begin{parts}
		\part Prove that $G$ is a group with respect to multiplication. Matrix
		multiplication is always associative, so you may assume that. But check
		closure and the existence of an identity element and inverse elements
		very carefully.
		\begin{solution}
			\begin{enumerate}
				\item Given any two elements in $G$, we check whether their product is also in $G$. The determinant of the product of two matrices is the product of their determinants. Since neither determinant is equal to $0$, the product will also not be. Since $\Q$ is closed under multiplication, $G$ is also closed under multiplication.
				\item The identity element is the standard identity matrix (ones on the diagonal and zeros elsewhere).
				\item Every element in $G$ has an inverse, as defined by linear algebra. The determinant of the inverse will be the inverse of the determinant of the original matrix. Therefore, the determinant is a rational number not equal to zero, so it is in $G$.
			\end{enumerate}
		\end{solution}

		\part A group is called \textbf{Abelian} iff its operation is
		commutative. Is this group Abelian?
		\begin{solution}
			No. Consider the matrices
		\[ A = \left(\begin{array}{cc} 1 & 1 \\ 2 & 3 \end{array}\right) \]
			and
		\[ B = \left(\begin{array}{cc} 2 & 3 \\ 3 & 4 \end{array}\right) \]
			The determinant of $A$ is $1$, and the determinant of $B$ is 2. Their products are
		\[ A \cdot B = \left(\begin{array}{cc} 5 & 7 \\ 13 & 18 \end{array}\right) \]
			or
		\[ B \cdot A = \left(\begin{array}{cc} 8 & 11 \\ 11 & 15 \end{array}\right) \]
			Therefore $A \cdot B \not= B \cdot A$. Therefore $G$ is non-Abelian.
		\end{solution}
	\end{parts}

	\question 
	Prove: If $(G, \star)$ is a group, $a \in G$, and $a \star b = b$ for some $b \in G$, then $a$ is the identity element of $G$.
	\begin{solution}
		Let $e$ be the identity element.
		\begin{align*}
			             a \star b &= b \\
			a \star b \star b^{-1} &= b \star b^{-1} \\
			             a \star e &= e \\
			                     a &= e
		\end{align*}
	\end{solution}

\end{questions}
\end{document}
