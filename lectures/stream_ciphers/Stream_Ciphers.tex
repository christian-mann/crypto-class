\documentclass[12pt]{beamer}
%\documentclass[handout,xcolor=pdflatex,dvipsnames,table,12pt]{beamer}
\usepackage[latin1]{inputenc}
%\usepackage[T1]{fontenc}
\usepackage{amsmath} % for math AMS fonts
\usepackage{graphicx} % to include figures
\usepackage{subfigure} % to have figures in figures
\usepackage{multimedia} % to include movies
\usepackage{listings} % to display code
\usepackage{colortbl} % colored tables
\usepackage[latin1]{inputenc} % support for accented letters, etc.
\usepackage{amsthm}
\usepackage{hyperref}
\usepackage{ulem}
\usepackage{fixltx2e}

\usetheme{Warsaw}
\setbeamercovered{transparent}

\title[Introduction to Cryptography]{Stream Ciphers}
\author{Chad Johnson -- chad-johnson@utulsa.edu}
\institute{University of Tulsa\\
Tulsa, Oklahoma 74104}
\date{\today}

\logo{\includegraphics[height=1.5cm]{pictures/SFSLogoMain}}

\begin{document}

\lstset{
language=python,                % choose the language of the code
%basicstyle=\footnotesize,       % the size of the fonts that are used for the code
%numbers=left,                   % where to put the line-numbers
%numberstyle=\footnotesize,      % the size of the fonts that are used for the line-numbers
%stepnumber=2,                   % the step between two line-numbers. If it's 1 each line will be numbered
%%umbersep=5pt,                  % how far the line-numbers are from the code
%backgroundcolor=\color{white},  % choose the background color. You must add \usepackage{color}
showspaces=false,               % show spaces adding particular underscores
showstringspaces=false,         % underline spaces within strings
showtabs=false,                 % show tabs within strings adding particular underscores
%%frame=single,	                % adds a frame around the code
tabsize=4,	                % sets default tabsize to 2 spaces
%%captionpos=b,                   % sets the caption-position to bottom
%%breaklines=true,                % sets automatic line breaking
%%breakatwhitespace=false,        % sets if automatic breaks should only happen at whitespace
%%title=\lstname,                 % show the filename of files included with \lstinputlisting; also try caption instead of title
%escapeinside={\%*}{*)}          % if you want to add a comment within your code
%morekeywords={*,...}            % if you want to add more keywords to the set
}

\newtheorem{mydef}{Definition}

\begin{frame}
\titlepage
\end{frame}

\begin{frame}{Stream Ciphers}{}
\begin{block}{The loose definition}
A stream cipher is a symmetric key cipher where plaintext bits are combined with a pseudorandom cipher keystream to produce a stream of ciphertext bits.
\begin{itemize}
	\item Key point: The cipher keystream must not be recoverable.
\end{itemize}
\end{block}
\end{frame}

\begin{frame}{Stream Ciphers}{The operation}
\begin{block}{}
\begin{itemize}
	\item Each plaintext bit is operated on one at a time with the keystream bit
	\item Example
	\begin{itemize}
		\item C\textsubscript i = P\textsubscript i \textless oper\textgreater  K\textsubscript i
		\item C\textsubscript i+1 = P\textsubscript i+1 \textless oper\textgreater  K\textsubscript i+1
		\item C\textsubscript i+n = P\textsubscript i+n \textless oper\textgreater  K\textsubscript i+n
		\item where \textless oper\textgreater is whatever operations are used in the stream cipher algorithm
	\end{itemize}
	\item Operation is commonly XOR
\end{itemize}
\end{block}
\end{frame}

\begin{frame}{Stream Ciphers}{One Time Pad}
\begin{block}{Vernam Cipher vs Stream Cipher}
Very similar to a one time pad. However, huge differences.
\begin{itemize}
	\item One Time Pad keystream is:
	\begin{itemize}
		\item at least as long as the plaintext to encrypt
		\item ideally completely random
		\item never used more than once
	\end{itemize}
	\item Stream Cipher keystream is:
	\begin{itemize}
		\item often a fixed size, e.g. 128-bits
		\item pseudorandom 
		\item repeated both during the stream and potentially in later encryptions
	\end{itemize}
\end{itemize}
\end{block}
\end{frame}

\begin{frame}{Stream Ciphers}{Popular Types}
\begin{block}{Synchronous Stream Cipher}
\begin{itemize}
	\item Keystream generated separately from plaintext and ciphertext
	\item Sender and receiver must stay in sync
	\item Added or dropped bits from message causes two ends to lose sync
	\begin{itemize}
		\item Causing complete loss of ability to decrypt
	\end{itemize}
	\item Single bit errors in transmission result in only a single bit decryption error
\end{itemize}
\end{block}
\end{frame}

\begin{frame}{Stream Ciphers}{Popular Types}
\begin{block}{Self-synchronizing Stream Cipher}
\begin{itemize}
	\item Keystream computed from previous N ciphertext bits
	\item ``Self-synchronizing'' meaning the sender and receiver can sync back up in case of added or removed bits
	\item Single bit errors in transmission result in N plaintext bits
	\item Examples: Cipher Feedback (CFB) and Output Feedback (OFB) modes
\end{itemize}
\end{block}
\end{frame}

\begin{frame}{Stream Ciphers}{Liner Feedback Shift Registers}
\begin{block}{LFSRs}
\begin{itemize}
	\item Shift register whose input is a linear function of it's previous state
	\begin{itemize}
		\item Most common LFSR: XOR
	\end{itemize}
	\item Easy to implement in hardware
	\item Easy to validate mathematically
	\item WRT Stream Ciphers, not great protection
\end{itemize}
\end{block}
\end{frame}

\begin{frame}{Stream Ciphers}{Improvements on LFSRs}
\begin{block}{Techniques to improve upon LFSRs}
\begin{itemize}
	\item Non-linear combining functions 
	\begin{itemize}
		\item Set the output of several parallel LFSRs as input to a non-linear Boolean combining function
	\end{itemize}
	\item Clock-controlled generators
	\begin{itemize}
		\item LFSR clocked at irregular intervals
		\item Couple different techniques
	\end{itemize}
	\item Filter generators
	\begin{itemize}
		\item To improve security, pass entire state of LFSR into a non-linear filtering function
	\end{itemize}
\end{itemize}
\end{block}
\end{frame}

\begin{frame}{Stream Ciphers}{Security Considerations}
\begin{block}{What are things to consider?}
\begin{itemize}
	\item Large keystream
	\item Impossible to recover keystream or initial state of keystream generation
	\item No biases toward 0 or 1, for instance to look like noise
	\item No relationships between keystream and ciphertext or plaintext
	\begin{itemize}
		\item To provide protection against known *text attacks
	\end{itemize}
	\item Ideally, no weak keys (the dream!)
\end{itemize}
\end{block}
\end{frame}

\begin{frame}{Stream Ciphers}{Common Protocols}
\begin{block}{Stream Ciphers in Use}
\begin{itemize}
	\item RC4 - hands down, the most common
	\begin{itemize}
		\item Used in WEP, SSL, TLS
	\end{itemize}
	\item A5/1, A5/2 - GSM encryption
	\item Helix/Phelix, ISAAC, Panama, ...
\end{itemize}
\end{block}
\begin{block}{The end result?}
\begin{itemize}
	\item All are claimed to be insecure.
	\item Some actually proven to be insecure.
\end{itemize}
\end{block}
\end{frame}

\begin{frame}{Stream Ciphers}{Attacks}
\begin{block}{Key Reuse Attack}
\begin{itemize}
	\item Alice uses the same key to messages to both Bob and Charlie
	\item Possible to extract the key by performing operations on both messages to Bob and Charlie
	\begin{itemize}
		\item Especially true if XOR is the only means to encrypt the message
	\end{itemize}
	\item One solution is to use an IV to create a one-time key, such as in WEP and WPA.
	\begin{itemize}
		\item The issue with WEP is the key was only 24 bits long, not nearly long enough.
	\end{itemize}
\end{itemize}
\end{block}
\end{frame}

\begin{frame}{Stream Ciphers}{Attacks}
\begin{block}{Bit-flipping Attack}
\begin{itemize}
	\item Adversary can modify known fields to influence the ciphertext and when decrypted, control the plaintext
	\item One remediation is to use a MAC to detect tampering
\end{itemize}
\end{block}
\begin{block}{Correlation Attack}
\begin{itemize}
	\item Possible if a strong correlation can be identified WRT individual inputs to LFSRs and output of combining functions
	\item If this can occur in a few different locations, you can ease the processing required to break the entire cipher
	\begin{itemize}
		\item Divide and Conquer algorithm! This should sound familiar.
	\end{itemize}
\end{itemize}
\end{block}
\end{frame}

\begin{frame}{Questions / Comments?}{}
\end{frame}

\end{document}
