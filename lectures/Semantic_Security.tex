\documentclass[12pt]{beamer}
%\documentclass[handout,xcolor=pdflatex,dvipsnames,table,12pt]{beamer}
\usepackage[latin1]{inputenc}
%\usepackage[T1]{fontenc}
\usepackage{amsmath} % for math AMS fonts
\usepackage{graphicx} % to include figures
\usepackage{subfigure} % to have figures in figures
\usepackage{multimedia} % to include movies
\usepackage{listings} % to display code
\usepackage{colortbl} % colored tables
\usepackage[latin1]{inputenc} % support for accented letters, etc.
\usepackage{amsthm}
\usepackage{hyperref}
\usepackage{ulem}

\usetheme{Warsaw}
\setbeamercovered{transparent}

\title[Introduction to Cryptography]{Semantic Security}
\author{Christian Mann -- christian-mann@utulsa.edu}
\institute{University of Tulsa\\
Tulsa, Oklahoma 74104}
\date{\today}

\logo{\includegraphics[height=1.5cm]{pictures/SFSLogoMain}}

\begin{document}

\lstset{
language=python,                % choose the language of the code
%basicstyle=\footnotesize,       % the size of the fonts that are used for the code
%numbers=left,                   % where to put the line-numbers
%numberstyle=\footnotesize,      % the size of the fonts that are used for the line-numbers
%stepnumber=2,                   % the step between two line-numbers. If it's 1 each line will be numbered
%%umbersep=5pt,                  % how far the line-numbers are from the code
%backgroundcolor=\color{white},  % choose the background color. You must add \usepackage{color}
showspaces=false,               % show spaces adding particular underscores
showstringspaces=false,         % underline spaces within strings
showtabs=false,                 % show tabs within strings adding particular underscores
%%frame=single,	                % adds a frame around the code
tabsize=4,	                % sets default tabsize to 2 spaces
%%captionpos=b,                   % sets the caption-position to bottom
%%breaklines=true,                % sets automatic line breaking
%%breakatwhitespace=false,        % sets if automatic breaks should only happen at whitespace
%%title=\lstname,                 % show the filename of files included with \lstinputlisting; also try caption instead of title
%escapeinside={\%*}{*)}          % if you want to add a comment within your code
%morekeywords={*,...}            % if you want to add more keywords to the set
}

\newtheorem{mydef}{Definition}


\begin{frame}
\titlepage
\end{frame}

\begin{frame}{Mathematical Preliminaries}
\end{frame}

\end{document}
