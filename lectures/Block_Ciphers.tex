\documentclass[12pt]{beamer}
%\documentclass[handout,xcolor=pdflatex,dvipsnames,table,12pt]{beamer}
\usepackage[latin1]{inputenc}
%\usepackage[T1]{fontenc}
\usepackage{amsmath} % for math AMS fonts
\usepackage{graphicx} % to include figures
\usepackage{subfigure} % to have figures in figures
\usepackage{multimedia} % to include movies
\usepackage{listings} % to display code
\usepackage{colortbl} % colored tables
\usepackage[latin1]{inputenc} % support for accented letters, etc.
\usepackage{amsthm}
\usepackage{hyperref}
\usepackage{ulem}

\usetheme{Warsaw}
\setbeamercovered{transparent}

\title[Class Name]{Block Ciphers}
\author{Chad Johnson -- chad-johnson@utulsa.edu}
\institute{University of Tulsa\\
Tulsa, Oklahoma 74104}
\date{\today}

\logo{\includegraphics[height=1.5cm]{pictures/SFSLogoMain}}

\begin{document}

\lstset{
language=python,                % choose the language of the code
%basicstyle=\footnotesize,       % the size of the fonts that are used for the code
%numbers=left,                   % where to put the line-numbers
%numberstyle=\footnotesize,      % the size of the fonts that are used for the line-numbers
%stepnumber=2,                   % the step between two line-numbers. If it's 1 each line will be numbered
%%umbersep=5pt,                  % how far the line-numbers are from the code
%backgroundcolor=\color{white},  % choose the background color. You must add \usepackage{color}
showspaces=false,               % show spaces adding particular underscores
showstringspaces=false,         % underline spaces within strings
showtabs=false,                 % show tabs within strings adding particular underscores
%%frame=single,	                % adds a frame around the code
tabsize=4,	                % sets default tabsize to 2 spaces
%%captionpos=b,                   % sets the caption-position to bottom
%%breaklines=true,                % sets automatic line breaking
%%breakatwhitespace=false,        % sets if automatic breaks should only happen at whitespace
%%title=\lstname,                 % show the filename of files included with \lstinputlisting; also try caption instead of title
%escapeinside={\%*}{*)}          % if you want to add a comment within your code
%morekeywords={*,...}            % if you want to add more keywords to the set
}

\newtheorem{mydef}{Definition}


\begin{frame}
\titlepage
\end{frame}


% no outline

% note, you should have three sections maximum.  two is good.  subsubsections are evil.
% new slides begin with teh \begin{frame} and end with \end{frame}

\iffalse
Generic Block Cipher description
Block cipher modes, at least 4.
algorithms
weaknesses
how to break a block cipher
\fi
\begin{frame}{Block Cipher}{}
\begin{block}{The formal definition}
``In cryptography, a block cipher is a deterministic algorithm operating on fixed-length groups of bits, called blocks, with an unvarying transformation that is specified by a symmetric key.''
\end{block}

\begin{block}{The informal definition}
A cipher that has the same key to encrypt and decrypt.
\begin{itemize}
	\item There are exceptions
\end{itemize}
\end{block}
\end{frame}

\begin{frame}{Block Cipher}{Generic Description}
\begin{block}{}
Generically, a string of text encrypted using a block cipher will always produce the same output ciphertext.
\begin{itemize}
	\item This leads to multiple problems with regards to security
	\item Block replay is a real problem
\end{itemize}
This can be corrected with the use of an initialization vector
\begin{itemize}
	\item Usually created by a random number generator
\end{itemize}
\end{block}
\end{frame}

\begin{frame}{Block Ciphers}{What you need to know}
\begin{block}{Elements of the cipher that must be agreed upon}
\begin{itemize}
	\item block size
	\item key size
	\item algorithm
	\begin{itemize}
		\item Steps to the dance
	\end{itemize}
	\item in some cases, initialization vector
\end{itemize}
\end{block}
\end{frame}

\begin{frame}{Block Ciphers}{The modes}
\begin{block}{Four commonly used modes}
\begin{itemize}
	\item Electronic Code Book (ECB)
	\item Cipher Block Chaining (CBC)
	\item Cipher Feedback (CFB)
	\item Output Feedback (OFB)
\end{itemize}
\end{block}
\end{frame}

\begin{frame}{Electronic Code Book}{Overview}
\begin{block}{ECB}
\begin{itemize}
	\item Simplest form of encryption
	\item Each block of data is encrypted independently
	\item Errors do not propagate from block to block
	\item Really easy to break
	\item Susceptible to replay attacks
\end{itemize}
\end{block}
\end{frame}

\begin{frame}{Electronic Code Book}{Encryption}
\begin{center}
\pgfimage[height = 4 cm]{pictures/ecbModeEnc}
\end{center}
\end{frame}

\begin{frame}{Electronic Code Book}{Decryption}
\begin{center}
\pgfimage[height = 4 cm]{pictures/ecbModeDec}
\end{center}
\end{frame}

\begin{frame}{Electronic Code Book}{Example}
\begin{center}
\pgfimage[height = 4 cm]{pictures/ecbEncryptedImage}
\end{center}
\begin{block}{}
An example of ECB encrypting each pixel of the image on the left to the image in the middle. 
\begin{itemize}
	\item The item on the right is encrypted with any of the other modes
\end{itemize}
\end{block}

\end{frame}

\begin{frame}{Cipher Block Chaining}{Overview}
\begin{block}{CBC}
\begin{itemize}
	\item Improvement upon ECB which uses an IV to encrypt the first block
	\item Each subsequent block uses the previous ciphertext in place of the IV
	\begin{itemize}
		\item This means encryption cannot be parallelized
	\end{itemize}
	\item Decryption only uses the IV to decrypt the first block
	\item Subsequent blocks are decrypted using two adjacent ciphertext blocks
	\begin{itemize}
		\item This means decryption can be parallelized
	\end{itemize}
\end{itemize}
\end{block}
\end{frame}

\begin{frame}{Cipher Block Chaining}{Encrypt}
\begin{center}
\pgfimage[height = 4 cm]{pictures/cbcModeEnc}
\end{center}
\end{frame}

\begin{frame}{Cipher Block Chaining}{Decrypt}
\begin{center}
\pgfimage[height = 4 cm]{pictures/cbcModeDec}
\end{center}
\end{frame}

\begin{frame}{Block Cipher Hazards}{A ``simple'' attack}
\begin{block}{Padding Oracle Attack}
\begin{itemize}
	\item Both ECB and CBC are susceptible
	\item Occurs when a server leaks data detailing if a message is correctly padded
	\begin{itemize}
		\item Can be used to decrypt messages using the Oracle's key without the Oracle knowing
	\end{itemize}
	\item Couple popular variants released the last few years
	\begin{itemize}
		\item Lucky Thirteen
		\item POODLE
	\end{itemize}
\end{itemize}
\end{block}
\end{frame}

\begin{frame}{Cipher Feedback}{Overview}
\begin{block}{CFB}
\begin{itemize}
	\item Effectively makes a stream cipher out of a block cipher
	\item Two major improvements over CBC
	\begin{itemize}
		\item No need for padding
		\item Only ever encrypt. No true decryption routine required
	\end{itemize}
	\item Similar to CBC: decryption can be parallelized but encryption cannot
	\item An error in 1 bit during decryption will affect that bit plus the entire next block
\end{itemize}
\end{block}
\end{frame}

\begin{frame}{Cipher Feedback}{Encrypt}
\begin{center}
\pgfimage[height = 4 cm]{pictures/cfbModeEnc}
\end{center}
\end{frame}

\begin{frame}{Cipher Feedback}{Decrypt}
\begin{center}
\pgfimage[height = 4 cm]{pictures/cfbModeDec}
\end{center}
\end{frame}

\begin{frame}{Output Feedback}{Overview}
\begin{block}{OFB}
\begin{itemize}
	\item Another block cipher acted as a stream cipher
	\item Uses the power of XOR symmetry as both encryption and decryption are identical operations
	\item Neither encryption nor decryption can be performed in parallel
	\begin{itemize}
		\item Slight exception in that some of the decryption work can be performed while waiting on proper input
	\end{itemize}
\end{itemize}
\end{block}
\end{frame}

\begin{frame}{Output Feedback}{Encrypt}
\begin{center}
\pgfimage[height = 4 cm]{pictures/ofbModeEnc}
\end{center}
\end{frame}

\begin{frame}{Output Feedback}{Decrypt}
\begin{center}
\pgfimage[height = 4 cm]{pictures/ofbModeDec}
\end{center}
\end{frame}

%padding oracle attacks

\begin{frame}{Questions / Comments?}{}
\end{frame}

\end{document}
