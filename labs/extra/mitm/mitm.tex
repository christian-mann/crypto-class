\newif\ifinstructornotes
%\instructornotestrue % comment out to hide instructor notes

\documentclass[11pt,oneside]{article}
\usepackage{geometry}
\usepackage[T1]{fontenc}
\usepackage{lastpage}
\usepackage{fancyhdr}
\usepackage{hyperref}
\usepackage{graphicx}
\usepackage{listings} % to display code
\usepackage{parskip}
\usepackage{url}
\pagestyle{fancy}
%\geometry{letterpaper,tmargin=.75in,bmargin=1.25in,lmargin=.75in,rmargin=.75in,headheight=13.6pt,headsep=0in,footskip=.3in}
\geometry{letterpaper,tmargin=.5in,bmargin=.5in,lmargin=.5in,rmargin=.5in,headheight=13.6pt,headsep=0in,footskip=.3in}

%%%%%%%%%%%%%%%%%%%%%%%%%%%%%%%%%%%%%%%%%%%%%%%%%%%%%%%%%%%%%%%%%%%%%%%%%%%%%%%%

\setlength{\parindent}{0in}
%\setlength{\parskip}{0.0in}
%\setlength{\itemsep}{0in}
%\setlength{\topsep}{0in}
%\setlength{\tabcolsep}{0in}

%%%%%%%%%%%%%%%%%%%%%%%%%%%%%%%%%%%%%%%%%%%%%%%%%%%%%%%%%%%%%%%%%%%%%%%%%%%%%%%%

\newcommand{\class}{Introduction to Cryptography}
\newcommand{\project}{Lab 8 Candidate:\\Meet In The Middle Attack on Block
Cipher}

\renewcommand{\headrulewidth}{0pt}
%\fancyfoot{}
%\cfoot{\quad \quad \quad \quad \quad \quad \thepage\ of \pageref{LastPage}}
% quads to fix center...
\cfoot{}

%%%%%%%%%%%%%%%%%%%%%%%%%%%%%%%%%%%%%%%%%%%%%%%%%%%%%%%%%%%%%%%%%%%%%%%%%%%%%%%%

% Style
\newcommand{\sectionfont}{phv} % Helvetica
\newcommand{\bodyfont}{ppl} % Palatino

\renewcommand{\section}[1] {
    \vspace{12pt}{\quad\fontfamily{\sectionfont}\selectfont\Large\scshape\textbf{#1}}\\[-10pt]
    \vspace{8pt}\rule{\textwidth}{1pt}\\[-16pt]

    % this space is needed
}

\renewcommand{\subsection}[1] {
    \vspace{12pt}{\fontfamily{\sectionfont}\selectfont\large\scshape\textbf{#1}}\\[-10pt]
    %\vspace{8pt}\rule{\textwidth}{1pt}\\[-16pt]

    % this space is needed
}


% \newcommand{\project}[2] {
%                 \begin{tabular}{p{.25\linewidth}p{.05\linewidth}p{.7\linewidth}}
%                         \textbf{#1} & & #2 \\\\
%                 \end{tabular}
% 
%         % this space is needed
% }

%%%%%%%%%%%%%%%%%%%%%%%%%%%%%%%%%%%%%%%%%%%%%%%%%%%%%%%%%%%%%%%%%%%%%%%%%%%%%%%%

\lstset{
language=python,                % choose the language of the code
%basicstyle=\footnotesize,       % the size of the fonts that are used for the code
%numbers=left,                   % where to put the line-numbers
%numberstyle=\footnotesize,      % the size of the fonts that are used for the line-numbers
%stepnumber=2,                   % the step between two line-numbers. If it's 1 each line will be numbered
%%umbersep=5pt,                  % how far the line-numbers are from the code
%backgroundcolor=\color{white},  % choose the background color. You must add \usepackage{color}
showspaces=false,               % show spaces adding particular underscores
showstringspaces=false,         % underline spaces within strings
showtabs=false,                 % show tabs within strings adding particular underscores
%%frame=single,	                % adds a frame around the code
tabsize=4,	                % sets default tabsize to 2 spaces
%%captionpos=b,                   % sets the caption-position to bottom
%%breaklines=true,                % sets automatic line breaking
%%breakatwhitespace=false,        % sets if automatic breaks should only happen at whitespace
%%title=\lstname,                 % show the filename of files included with \lstinputlisting; also try caption instead of title
%escapeinside={\%*}{*)}          % if you want to add a comment within your code
%morekeywords={*,...}            % if you want to add more keywords to the set
}


\begin{document}

\fontfamily{\bodyfont} \selectfont \small
\thispagestyle{empty}
\begin{center}
    \fontfamily{\sectionfont}\selectfont\huge\scshape\textbf{\class}
\end{center}
\begin{center}
    \fontfamily{\sectionfont}\selectfont\large\scshape\textbf{\project}
\end{center}

\section{Overview}

In this lab, you will exploit a toy cryptosystem that uses a
substitution-permutation network. The key for this system is 6 bytes long, split
into two 3-byte subkeys. The system uses double encryption, wherein the
plaintext is encrypted with one subkey, and then that is encrypted with the
second subkey.

This is vulnerable to a meet-in-the-middle attack, also called a time-memory
tradeoff. You will write a script to perform this attack.

2DES is the most well-known system wherein this sort of attack is possible, but
even today, attacks can take more than an hour. I don't want you do have to deal
with that, so we use a fake system.

\section{Assignment}

You will be implementing the majority of the functionality for this lab -- both
server and client. I will give you the base structure that I want you to follow,
as well as the core encryption code, but I will ask you to implement the
majority. You are of course welcome and encouraged to use the library that I've
given you, as well as code that you wrote.

\ifinstructornotes
\textbf{Instructor Note:} This comes from PicoCTF 2014.
\fi

Come up with a flag. I encourage creativity here. This flag is very freeform; it
can be any bytes, of pretty much any structure. Prepend this flag with
``message:'', an 8-byte string. This string is precisely one block long.
Generate a random 6-byte key, and encrypt the combined plaintext.

Encrypt the known plaintext under every possible key, and store the output.
Probably in a dictionary or similar. Then, decrypt the known ciphertext under
every possible key. One pair of keys will produce identical intermediate
results. This is the correct key.

\section{Rules}

Your code should resemble the attacks that I have given you before.
Particularly, you are required to use Pyro4 for remote object calls, if you have
any. You may distribute the flag already-encrypted, or make it a constant method
on a remote object, as in previous labs.

Please do not make the flag too offensive. I spend a lot of time on the
Internet, and not a lot can shock me, but there's no point in being senselessly
shocking. As I said, I encourage creativity.

\section{Grading}
Labs are due Thursday, November 12 at 11:59pm. You will need to turn in
all code, including a complete server, example client, and working solution.

\end{document}
