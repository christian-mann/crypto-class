\newif\ifinstructornotes
%\instructornotestrue % comment out to hide instructor notes

\documentclass[11pt,oneside]{article}
\usepackage{geometry}
\usepackage[T1]{fontenc}
\usepackage{lastpage}
\usepackage{fancyhdr}
\usepackage{hyperref}
\usepackage{graphicx}
\usepackage{listings} % to display code
\usepackage{parskip}
\usepackage{url}
\pagestyle{fancy}
%\geometry{letterpaper,tmargin=.75in,bmargin=1.25in,lmargin=.75in,rmargin=.75in,headheight=13.6pt,headsep=0in,footskip=.3in}
\geometry{letterpaper,tmargin=.5in,bmargin=.5in,lmargin=.5in,rmargin=.5in,headheight=13.6pt,headsep=0in,footskip=.3in}

%%%%%%%%%%%%%%%%%%%%%%%%%%%%%%%%%%%%%%%%%%%%%%%%%%%%%%%%%%%%%%%%%%%%%%%%%%%%%%%%

\setlength{\parindent}{0in}
%\setlength{\parskip}{0.0in}
%\setlength{\itemsep}{0in}
%\setlength{\topsep}{0in}
%\setlength{\tabcolsep}{0in}

%%%%%%%%%%%%%%%%%%%%%%%%%%%%%%%%%%%%%%%%%%%%%%%%%%%%%%%%%%%%%%%%%%%%%%%%%%%%%%%%

\newcommand{\class}{Introduction to Cryptography}
\newcommand{\project}{Lab 8 Candidate:\\Stream Cipher Repeated Key}

\renewcommand{\headrulewidth}{0pt}
%\fancyfoot{}
%\cfoot{\quad \quad \quad \quad \quad \quad \thepage\ of \pageref{LastPage}}
% quads to fix center...
\cfoot{}

%%%%%%%%%%%%%%%%%%%%%%%%%%%%%%%%%%%%%%%%%%%%%%%%%%%%%%%%%%%%%%%%%%%%%%%%%%%%%%%%

% Style
\newcommand{\sectionfont}{phv} % Helvetica
\newcommand{\bodyfont}{ppl} % Palatino

\renewcommand{\section}[1] {
    \vspace{12pt}{\quad\fontfamily{\sectionfont}\selectfont\Large\scshape\textbf{#1}}\\[-10pt]
    \vspace{8pt}\rule{\textwidth}{1pt}\\[-16pt]

    % this space is needed
}

\renewcommand{\subsection}[1] {
    \vspace{12pt}{\fontfamily{\sectionfont}\selectfont\large\scshape\textbf{#1}}\\[-10pt]
    %\vspace{8pt}\rule{\textwidth}{1pt}\\[-16pt]

    % this space is needed
}


% \newcommand{\project}[2] {
%                 \begin{tabular}{p{.25\linewidth}p{.05\linewidth}p{.7\linewidth}}
%                         \textbf{#1} & & #2 \\\\
%                 \end{tabular}
% 
%         % this space is needed
% }

%%%%%%%%%%%%%%%%%%%%%%%%%%%%%%%%%%%%%%%%%%%%%%%%%%%%%%%%%%%%%%%%%%%%%%%%%%%%%%%%

\lstset{
language=python,                % choose the language of the code
%basicstyle=\footnotesize,       % the size of the fonts that are used for the code
%numbers=left,                   % where to put the line-numbers
%numberstyle=\footnotesize,      % the size of the fonts that are used for the line-numbers
%stepnumber=2,                   % the step between two line-numbers. If it's 1 each line will be numbered
%%umbersep=5pt,                  % how far the line-numbers are from the code
%backgroundcolor=\color{white},  % choose the background color. You must add \usepackage{color}
showspaces=false,               % show spaces adding particular underscores
showstringspaces=false,         % underline spaces within strings
showtabs=false,                 % show tabs within strings adding particular underscores
%%frame=single,	                % adds a frame around the code
tabsize=4,	                % sets default tabsize to 2 spaces
%%captionpos=b,                   % sets the caption-position to bottom
%%breaklines=true,                % sets automatic line breaking
%%breakatwhitespace=false,        % sets if automatic breaks should only happen at whitespace
%%title=\lstname,                 % show the filename of files included with \lstinputlisting; also try caption instead of title
%escapeinside={\%*}{*)}          % if you want to add a comment within your code
%morekeywords={*,...}            % if you want to add more keywords to the set
}


\begin{document}

\fontfamily{\bodyfont} \selectfont \small
\thispagestyle{empty}
\begin{center}
    \fontfamily{\sectionfont}\selectfont\huge\scshape\textbf{\class}
\end{center}
\begin{center}
    \fontfamily{\sectionfont}\selectfont\large\scshape\textbf{\project}
\end{center}

\section{Overview}

In this lab, you will attack a misconfigured stream cipher. In particular, you
will attack CTR mode encryption, but with a key/nonce combination that is fixed
across multiple encryptions.

\section{Assignment}

\subsection{Part 1: Implementing CTR}

You will be implementing the majority of the functionality for this lab -- both
server and client. I will give you the base structure that I want you to follow,
but I will ask you to implement the majority. You are of course welcome and
encouraged to use the library that I've given you, as well as code that you
wrote.

If you don't already have CTR mode encryption code lying around, you will need
to find or create an implementation for it. Note: You will probably want to get
a pure implementation; do not use builtin libraries like OpenSSL.

\subsection{Part 2: Oracle}

Take your CTR encrypt/decrypt function and fix its nonce value. (to 0, if you'd
like. Note: do not fix the \textit{counter}, just the nonce.) Generate a random
AES key.

Obtain or create a list of around 40 humorous quips, texts, sayings, lines of
poetry, et cetera.  Each one should be around the same length, but not exactly.

Create an oracle that chooses a random plaintext and encrypts it with CTR mode
under your static key and static nonce. Alternatively, encrypt each one and hold
on to the ciphertexts.

\subsection{Part 3: Decryption}

Because the nonce was not randomized, each ciphertext has been encrypted against
the same encryption. This is Very Bad.

Remember that, like most stream ciphers (including RC4, and any block cipher in
CTR mode), the actual ``encryption'' of a byte of data boils down to a single
XOR operation.

Create a solution that solves this problem in the same way you solved
repeating-key XOR, in the first project. While it should be impossible to solve
for the AES key, you can solve for the keystream directly. Remember that some
ciphertexts are longer than others; you may need to be careful with truncation.
As in the first project, your solution will likely not arrive at a perfect
decryption, but it should get close.

\section{Rules}

Your code should resemble the labs that I have given you before. Particularly,
you are required to use Pyro4 for remote object calls. You may distribute the
flags already-encrypted, or make it a constant method on a remote object, as in
previous labs.

For this lab, I will require a fully automated solution, unlike Lab 1.

\section{Grading}
Labs are due Thursday, November 12 at 11:59pm. You will need to turn in
all code, including a complete server, example client, and working solution.

\end{document}
