\newif\ifinstructornotes
%\instructornotestrue % comment out to hide instructor notes

\documentclass[11pt,oneside]{article}
\usepackage{geometry}
\usepackage[T1]{fontenc}
\usepackage{lastpage}
\usepackage{fancyhdr}
\usepackage{hyperref}
\usepackage{graphicx}
\usepackage{listings} % to display code
\usepackage{parskip}
\usepackage{url}
\pagestyle{fancy}
%\geometry{letterpaper,tmargin=.75in,bmargin=1.25in,lmargin=.75in,rmargin=.75in,headheight=13.6pt,headsep=0in,footskip=.3in}
\geometry{letterpaper,tmargin=.5in,bmargin=.5in,lmargin=.5in,rmargin=.5in,headheight=13.6pt,headsep=0in,footskip=.3in}

%%%%%%%%%%%%%%%%%%%%%%%%%%%%%%%%%%%%%%%%%%%%%%%%%%%%%%%%%%%%%%%%%%%%%%%%%%%%%%%%

\setlength{\parindent}{0in}
%\setlength{\parskip}{0.0in}
%\setlength{\itemsep}{0in}
%\setlength{\topsep}{0in}
%\setlength{\tabcolsep}{0in}

%%%%%%%%%%%%%%%%%%%%%%%%%%%%%%%%%%%%%%%%%%%%%%%%%%%%%%%%%%%%%%%%%%%%%%%%%%%%%%%%

\newcommand{\class}{Introduction to Cryptography}
\newcommand{\project}{Lab 9 Candidate:\\Bleichenbacher's Attack}

\renewcommand{\headrulewidth}{0pt}
%\fancyfoot{}
%\cfoot{\quad \quad \quad \quad \quad \quad \thepage\ of \pageref{LastPage}}
% quads to fix center...
\cfoot{}

%%%%%%%%%%%%%%%%%%%%%%%%%%%%%%%%%%%%%%%%%%%%%%%%%%%%%%%%%%%%%%%%%%%%%%%%%%%%%%%%

% Style
\newcommand{\sectionfont}{phv} % Helvetica
\newcommand{\bodyfont}{ppl} % Palatino

\renewcommand{\section}[1] {
    \vspace{12pt}{\quad\fontfamily{\sectionfont}\selectfont\Large\scshape\textbf{#1}}\\[-10pt]
    \vspace{8pt}\rule{\textwidth}{1pt}\\[-16pt]

    % this space is needed
}

\renewcommand{\subsection}[1] {
    \vspace{12pt}{\fontfamily{\sectionfont}\selectfont\large\scshape\textbf{#1}}\\[-10pt]
    %\vspace{8pt}\rule{\textwidth}{1pt}\\[-16pt]

    % this space is needed
}


% \newcommand{\project}[2] {
%                 \begin{tabular}{p{.25\linewidth}p{.05\linewidth}p{.7\linewidth}}
%                         \textbf{#1} & & #2 \\\\
%                 \end{tabular}
% 
%         % this space is needed
% }

%%%%%%%%%%%%%%%%%%%%%%%%%%%%%%%%%%%%%%%%%%%%%%%%%%%%%%%%%%%%%%%%%%%%%%%%%%%%%%%%

\lstset{
language=python,                % choose the language of the code
%basicstyle=\footnotesize,       % the size of the fonts that are used for the code
%numbers=left,                   % where to put the line-numbers
%numberstyle=\footnotesize,      % the size of the fonts that are used for the line-numbers
%stepnumber=2,                   % the step between two line-numbers. If it's 1 each line will be numbered
%%umbersep=5pt,                  % how far the line-numbers are from the code
%backgroundcolor=\color{white},  % choose the background color. You must add \usepackage{color}
showspaces=false,               % show spaces adding particular underscores
showstringspaces=false,         % underline spaces within strings
showtabs=false,                 % show tabs within strings adding particular underscores
%%frame=single,	                % adds a frame around the code
tabsize=4,	                % sets default tabsize to 2 spaces
%%captionpos=b,                   % sets the caption-position to bottom
%%breaklines=true,                % sets automatic line breaking
%%breakatwhitespace=false,        % sets if automatic breaks should only happen at whitespace
%%title=\lstname,                 % show the filename of files included with \lstinputlisting; also try caption instead of title
%escapeinside={\%*}{*)}          % if you want to add a comment within your code
%morekeywords={*,...}            % if you want to add more keywords to the set
}


\begin{document}

\fontfamily{\bodyfont} \selectfont \small
\thispagestyle{empty}
\begin{center}
    \fontfamily{\sectionfont}\selectfont\huge\scshape\textbf{\class}
\end{center}
\begin{center}
    \fontfamily{\sectionfont}\selectfont\large\scshape\textbf{\project}
\end{center}

\section{Overview}

In this lab, you will attack a vulnerability with unpadded / unarmored RSA.
This should feel somewhat similar to a lab you performed earlier in the semester,
when you persuaded a server to encrypt the same message under three different
moduli, and by doing so were able to recover the plaintext. The attack is
described in a very important paper from Bleichenbacher in 1998.

The idea behind this attack is an oracle that decrypts a ciphertext and checks
the format of the plaintext against a particular format. With this idea, you can
create an adaptive chosen-ciphertext attack.

Look up the paper ``Chosen ciphertext attacks against protocols based on the RSA
encryption standard.'' You should obtain a paper by Bleichenbacher from CRYPTO
'98; you can probably find a .ps or .pdf version.

Read the paper. It describes a padding oracle attack on PKCS\#1v1.5. The attack
is similar in spirit to the CBC padding oracle you built earlier; it's an
``adaptive chosen ciphertext attack,'' which means you start with a valid
ciphertext and repeatedly corrupt it, bouncing the adulterated ciphertexts off
the target to learn things about the original.

This is a common flaw even in modern cryptosystems that use RSA.

It's also the most fun you can have building a crypto attack. It involves 9th
grade math, but also has you implementing an algorithm that is complex on par
with finding a minimum cost spanning tree.

\section{Assignment}

You will be implementing the majority of the functionality for this lab -- both
server and client. I will give you the base structure that I want you to follow,
but I will ask you to implement the majority. You are of course welcome and
encouraged to use the library that I've given you, as well as code that you
wrote.

\ifinstructornotes
\textbf{Instructor Note:} This is challenge 46 of the Matasano Crypto
Challenges.
\fi

Setup: \begin{itemize}
	\item Build an oracle function, just like you did in the padding oracle attack, but
		have it check for plaintext[0] == 0 and plaintext[1] == 2.
	\item Generate a 256 bit keypair; p and q each should be 128 bit primes.
	\item Plug d and n into your oracle function.
	\item PKCS1.5-pad a short message, like ``rsa padding'', and call it ``m''.
		Encrypt it to get ``c''.
	\item Decrypt ``c'' using your padding oracle.
\end{itemize}

For this lab, you are using an unrealistically small RSA modulus. You are only
targeting one specific step in the Bleichenbacher paper -- Step 2c, which
implements a fast, nearly O(log n) search for the plaintext.

As you read the paper, keep in mind:
\begin{itemize}
	\item RSA ciphertexts are just numbers.
	\item RSA is ``homomorphic'' with respect to multiplication, which means you
		can multiply c * RSA(2) to get a $c'$ that will decrypt to plaintext *
		2. Try multiplying ciphertexts with the RSA encryption of numbers so you
		know you understand it.
	\item What you need to grok for this challenge is that Bleichenbacher uses
		multiplication of ciphertexts the way the CBC oracle uses XORs of random
		blocks.
	\item A PKCS\#1v1.5 conformant plaintext, one that starts with 00:02, must
		be a number between 02:00:00...00 and 02:FF:FF...FF -- in other words,
		$2B$ and $3B-1$, where $B$ is the bit size of the modulus minus the
		first 16 bits. When you see $2B$ and $3B$, that's the idea the paper is
		working with.
\end{itemize}

To decrypt ``c'', you'll need Step 2a from the paper (the search for the first
``s'' that, when encrypted and multiplied with the ciphertext, produces a
conformant plaintext), Step 2c, the fast O(log n) search, and Step 3.

Your Step 3 code is probably not going to need to handle multiple ranges.

We recommend you just use the raw math from paper (check, check, double check
your translation to code) and not spend too much time trying to grok how the
math works.

\section{Rules}

Because this lab is fairly involved, I will not require you to create a remote
oracle, unless you want to. However, there should be an obvious separation
between ``server'' code and ``client'' code.

\section{Grading}
Labs are due Thursday, November 12 at 11:59pm. You will need to turn in
all code, including a complete server, example client, and working solution.

\end{document}
