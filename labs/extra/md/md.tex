\newif\ifinstructornotes
%\instructornotestrue % comment out to hide instructor notes

\documentclass[11pt,oneside]{article}
\usepackage{geometry}
\usepackage[T1]{fontenc}
\usepackage{lastpage}
\usepackage{fancyhdr}
\usepackage{hyperref}
\usepackage{graphicx}
\usepackage{listings} % to display code
\usepackage{parskip}
\usepackage{url}
\pagestyle{fancy}
%\geometry{letterpaper,tmargin=.75in,bmargin=1.25in,lmargin=.75in,rmargin=.75in,headheight=13.6pt,headsep=0in,footskip=.3in}
\geometry{letterpaper,tmargin=.5in,bmargin=.5in,lmargin=.5in,rmargin=.5in,headheight=13.6pt,headsep=0in,footskip=.3in}

%%%%%%%%%%%%%%%%%%%%%%%%%%%%%%%%%%%%%%%%%%%%%%%%%%%%%%%%%%%%%%%%%%%%%%%%%%%%%%%%

\setlength{\parindent}{0in}
%\setlength{\parskip}{0.0in}
%\setlength{\itemsep}{0in}
%\setlength{\topsep}{0in}
%\setlength{\tabcolsep}{0in}

%%%%%%%%%%%%%%%%%%%%%%%%%%%%%%%%%%%%%%%%%%%%%%%%%%%%%%%%%%%%%%%%%%%%%%%%%%%%%%%%

\newcommand{\class}{Introduction to Cryptography}
\newcommand{\project}{Lab 9 Candidate:\\Merkle-Damgard Multicollisions}

\renewcommand{\headrulewidth}{0pt}
%\fancyfoot{}
%\cfoot{\quad \quad \quad \quad \quad \quad \thepage\ of \pageref{LastPage}}
% quads to fix center...
\cfoot{}

%%%%%%%%%%%%%%%%%%%%%%%%%%%%%%%%%%%%%%%%%%%%%%%%%%%%%%%%%%%%%%%%%%%%%%%%%%%%%%%%

% Style
\newcommand{\sectionfont}{phv} % Helvetica
\newcommand{\bodyfont}{ppl} % Palatino

\renewcommand{\section}[1] {
    \vspace{12pt}{\quad\fontfamily{\sectionfont}\selectfont\Large\scshape\textbf{#1}}\\[-10pt]
    \vspace{8pt}\rule{\textwidth}{1pt}\\[-16pt]

    % this space is needed
}

\renewcommand{\subsection}[1] {
    \vspace{12pt}{\fontfamily{\sectionfont}\selectfont\large\scshape\textbf{#1}}\\[-10pt]
    %\vspace{8pt}\rule{\textwidth}{1pt}\\[-16pt]

    % this space is needed
}


% \newcommand{\project}[2] {
%                 \begin{tabular}{p{.25\linewidth}p{.05\linewidth}p{.7\linewidth}}
%                         \textbf{#1} & & #2 \\\\
%                 \end{tabular}
% 
%         % this space is needed
% }

%%%%%%%%%%%%%%%%%%%%%%%%%%%%%%%%%%%%%%%%%%%%%%%%%%%%%%%%%%%%%%%%%%%%%%%%%%%%%%%%

\lstset{
language=python,                % choose the language of the code
%basicstyle=\footnotesize,       % the size of the fonts that are used for the code
%numbers=left,                   % where to put the line-numbers
%numberstyle=\footnotesize,      % the size of the fonts that are used for the line-numbers
%stepnumber=2,                   % the step between two line-numbers. If it's 1 each line will be numbered
%%umbersep=5pt,                  % how far the line-numbers are from the code
%backgroundcolor=\color{white},  % choose the background color. You must add \usepackage{color}
showspaces=false,               % show spaces adding particular underscores
showstringspaces=false,         % underline spaces within strings
showtabs=false,                 % show tabs within strings adding particular underscores
%%frame=single,	                % adds a frame around the code
tabsize=4,	                % sets default tabsize to 2 spaces
%%captionpos=b,                   % sets the caption-position to bottom
%%breaklines=true,                % sets automatic line breaking
%%breakatwhitespace=false,        % sets if automatic breaks should only happen at whitespace
%%title=\lstname,                 % show the filename of files included with \lstinputlisting; also try caption instead of title
%escapeinside={\%*}{*)}          % if you want to add a comment within your code
%morekeywords={*,...}            % if you want to add more keywords to the set
}


\begin{document}

\fontfamily{\bodyfont} \selectfont \small
\thispagestyle{empty}
\begin{center}
    \fontfamily{\sectionfont}\selectfont\huge\scshape\textbf{\class}
\end{center}
\begin{center}
    \fontfamily{\sectionfont}\selectfont\large\scshape\textbf{\project}
\end{center}

\section{Overview}

In this lab, you will create a simple, nontrivial hash function, and then
generate a multitude of collisions for it.

The major feature you want in a hash function is collision resistance. It should
be hard to generate collisions, and it should be \textit{really} hard to collide
with a given hash (aka preimage).

Iterated hash functions have a problem: the effort to generate collisions scales
sublinearly. What's an iterated hash function? I'm talking about the
Merkle-Damgard construction, like this:
\begin{verbatim}
function MD(H, M, C):
  for M[i] in pad(M):
    H := C(M[i], H)
  return H
\end{verbatim}

For message M, initial state H, and compression function C.

This is the structure that SHA-1 and MD4 use. You can use this formula to build
a two-bit hash function out of spare crypto primitives you have lying around,
such as C = AES-128.

The cost of collisions scales sublinearly. What does that mean? If it's feasible
to find one collision, it's probably feasible to find a lot.

\section{Assignment}

I will give you the base structure that I want you to follow, but I will ask you
to implement the majority. You are of course welcome and encouraged to use the
library that I've given you, as well as code that you wrote.

\ifinstructornotes
\textbf{Instructor Note:} This is challenge 52 of the Matasano Crypto
Challenges.
\fi

To generate a lot of collisions: For a given state H, find two blocks that
collide. Now take the resulting hash from this collision as your new H and
repeat. Recognize that with each iteration you can actually double your
collisions by subbing in either of the two blocks for that slot.

Finding two collisions should takes $2^{b/2}$ work (where b is the bit-size of the
hash function), by the birthday paradox. Then finding $2^n$ colliding messages
only takes $n * 2^{b/2}$ work.

\subsection{Part 1: Build-a-hash-function workshop}

\begin{enumerate}
	\item Take a fast block cipher and use it as C.
	\item Make H pretty small -- 16 or 32 bits is plenty. Pick some initial H.
	\item H is going to be the input key and output block from C. Because C
		probably wants a full block, you'll need to pad it on the way in and
		drop some bits on the way out.
\end{enumerate}

Now write the function $f(n)$ that will generate $2^n$ collisions of this hash
function.

\subsection{Part 2: Double Collisions}

People have tried to strengthen hash functions by concatenating them, like this:
\begin{enumerate}
	\item Take hash functions $f$ and $g$.
	\item Build $h$ such that $h(x) = f(x) \| g(x)$.
\end{enumerate}

The idea is that if collisions in $f$ cost $2^{b_1/2}$ and collisions in $g$
cost $2^{b_2/2}$, then collisions in $h$ should be very expensive at
$2^{(b_1+b_2)/2}$. But we can do better!

\begin{enumerate}
	\item Pick the cheaper hash function; suppose it's $f$.
	\item Generate $2^{b_2/2}$ messages that collide in $f$.
	\item There's a good chance that your message pool has a collision in $g$.
	\item If it doesn't, keep generating collisions until you find it.
\end{enumerate}

Build a (slightly) more expensive hash function to pair with the one you just
used. Find a pair of messages that collide under both functions. Measure the
total number of collisions generated.

\section{Grading}
Labs are due Thursday, November 12 at 11:59pm. You will need to turn in
all code.

\end{document}
