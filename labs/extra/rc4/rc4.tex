\newif\ifinstructornotes
%\instructornotestrue % comment out to hide instructor notes

\documentclass[11pt,oneside]{article}
\usepackage{geometry}
\usepackage[T1]{fontenc}
\usepackage{lastpage}
\usepackage{fancyhdr}
\usepackage{hyperref}
\usepackage{graphicx}
\usepackage{listings} % to display code
\usepackage{parskip}
\usepackage{url}
\pagestyle{fancy}
%\geometry{letterpaper,tmargin=.75in,bmargin=1.25in,lmargin=.75in,rmargin=.75in,headheight=13.6pt,headsep=0in,footskip=.3in}
\geometry{letterpaper,tmargin=.5in,bmargin=.5in,lmargin=.5in,rmargin=.5in,headheight=13.6pt,headsep=0in,footskip=.3in}

%%%%%%%%%%%%%%%%%%%%%%%%%%%%%%%%%%%%%%%%%%%%%%%%%%%%%%%%%%%%%%%%%%%%%%%%%%%%%%%%

\setlength{\parindent}{0in}
%\setlength{\parskip}{0.0in}
%\setlength{\itemsep}{0in}
%\setlength{\topsep}{0in}
%\setlength{\tabcolsep}{0in}

%%%%%%%%%%%%%%%%%%%%%%%%%%%%%%%%%%%%%%%%%%%%%%%%%%%%%%%%%%%%%%%%%%%%%%%%%%%%%%%%

\newcommand{\class}{Introduction to Cryptography}
\newcommand{\project}{Lab 9 Candidate:\\RC4 Biases}

\renewcommand{\headrulewidth}{0pt}
%\fancyfoot{}
%\cfoot{\quad \quad \quad \quad \quad \quad \thepage\ of \pageref{LastPage}}
% quads to fix center...
\cfoot{}

%%%%%%%%%%%%%%%%%%%%%%%%%%%%%%%%%%%%%%%%%%%%%%%%%%%%%%%%%%%%%%%%%%%%%%%%%%%%%%%%

% Style
\newcommand{\sectionfont}{phv} % Helvetica
\newcommand{\bodyfont}{ppl} % Palatino

\renewcommand{\section}[1] {
    \vspace{12pt}{\quad\fontfamily{\sectionfont}\selectfont\Large\scshape\textbf{#1}}\\[-10pt]
    \vspace{8pt}\rule{\textwidth}{1pt}\\[-16pt]

    % this space is needed
}

\renewcommand{\subsection}[1] {
    \vspace{12pt}{\fontfamily{\sectionfont}\selectfont\large\scshape\textbf{#1}}\\[-10pt]
    %\vspace{8pt}\rule{\textwidth}{1pt}\\[-16pt]

    % this space is needed
}


% \newcommand{\project}[2] {
%                 \begin{tabular}{p{.25\linewidth}p{.05\linewidth}p{.7\linewidth}}
%                         \textbf{#1} & & #2 \\\\
%                 \end{tabular}
% 
%         % this space is needed
% }

%%%%%%%%%%%%%%%%%%%%%%%%%%%%%%%%%%%%%%%%%%%%%%%%%%%%%%%%%%%%%%%%%%%%%%%%%%%%%%%%

\lstset{
language=python,                % choose the language of the code
%basicstyle=\footnotesize,       % the size of the fonts that are used for the code
%numbers=left,                   % where to put the line-numbers
%numberstyle=\footnotesize,      % the size of the fonts that are used for the line-numbers
%stepnumber=2,                   % the step between two line-numbers. If it's 1 each line will be numbered
%%umbersep=5pt,                  % how far the line-numbers are from the code
%backgroundcolor=\color{white},  % choose the background color. You must add \usepackage{color}
showspaces=false,               % show spaces adding particular underscores
showstringspaces=false,         % underline spaces within strings
showtabs=false,                 % show tabs within strings adding particular underscores
%%frame=single,	                % adds a frame around the code
tabsize=4,	                % sets default tabsize to 2 spaces
%%captionpos=b,                   % sets the caption-position to bottom
%%breaklines=true,                % sets automatic line breaking
%%breakatwhitespace=false,        % sets if automatic breaks should only happen at whitespace
%%title=\lstname,                 % show the filename of files included with \lstinputlisting; also try caption instead of title
%escapeinside={\%*}{*)}          % if you want to add a comment within your code
%morekeywords={*,...}            % if you want to add more keywords to the set
}


\begin{document}

\fontfamily{\bodyfont} \selectfont \small
\thispagestyle{empty}
\begin{center}
    \fontfamily{\sectionfont}\selectfont\huge\scshape\textbf{\class}
\end{center}
\begin{center}
    \fontfamily{\sectionfont}\selectfont\large\scshape\textbf{\project}
\end{center}

\section{Overview}

This is a slightly more freeform lab than the other extra labs. In this lab, you
will experiment with RC4 keystream biases. In theory, you will implement a
ciphertext-only plaintext recovery attack, although it is possible it will be
too slow to reliably run.

RC4 is a stream cipher historically used in many places, such as WEP, HTTPS,
SSL, among others. One of its strengths is its speed. However, since its
creation, various biases have been identified in its keystream. These biases can
lead to partial or complete plaintext recovery, given enough encrypted text.

You will verify these biases exist, and implement a simple attack.

\section{Assignment}

You will implement the entirety of this lab -- both server and client. Due to
the resources required for the lab, I do not need you to actually use Pyro4 for
this lab. You are of course welcome and encouraged to use the library that I've
given you, as well as code that you wrote throughout the year.

\ifinstructornotes
\textbf{Instructor Note:} I made this challenge up all by myself!
\fi

You will find this paper useful: ``On the Security of RC4 in TLS and WPA'' by
AlFardan et al. It covers in detail the idea of keystream biases, example
attack scenarios, and their application to attacking real cryptosystems.

\subsection{Part 1: Observing Biases}

Once you have RC4 working, try to replicate the results you see in the paper.
Generate an RC4 key, and encrypt a keystream of 256 zero bytes (0x00). Repeat this
many, many times (probably at least 1 million), keeping track of the frequency
distributions for each byte of output. Make sure you can replicate the results
seen in the paper, at least somewhat.

Determine which bytes are biased the strongest, either towards or away from a
particular output.

Honestly, I have not done this myself, so I'm not sure that there is a strong
enough bias to show up in experiments like this. \textit{If you don't see a bias
after this}, modify the experiment. Insert a deliberate bias in your RC4
algorithm: Set the keystream byte to 0x00 randomly, based on the original key
and the original keystream byte.

\subsection{Part 2: Decrypting Ciphertext}

Generate a plaintext of 256 bytes. Encrypt it many times under random keys.
Determine the frequency distribution of each output byte whose corresponding
keystream had a strong bias. From this, you should be able to determine a
probability distribution of the original bytes.

Compare this probability distribution to the original plaintext.

\section{Grading}

Labs are due Thursday, November 12 at 11:59pm. You will need
to turn in all code. In addition, you will need to turn in a short report on
what you did, what you had to modify, and any supporting charts.

\end{document}
