\newif\ifinstructornotes
%\instructornotestrue % comment out to hide instructor notes

\documentclass[11pt,oneside]{article}
\usepackage{geometry}
\usepackage[T1]{fontenc}
\usepackage{lastpage}
\usepackage{fancyhdr}
\usepackage{hyperref}
\usepackage{graphicx}
\usepackage{listings} % to display code
\usepackage{parskip}
\usepackage{url}
\pagestyle{fancy}
%\geometry{letterpaper,tmargin=.75in,bmargin=1.25in,lmargin=.75in,rmargin=.75in,headheight=13.6pt,headsep=0in,footskip=.3in}
\geometry{letterpaper,tmargin=.5in,bmargin=.5in,lmargin=.5in,rmargin=.5in,headheight=13.6pt,headsep=0in,footskip=.3in}

%%%%%%%%%%%%%%%%%%%%%%%%%%%%%%%%%%%%%%%%%%%%%%%%%%%%%%%%%%%%%%%%%%%%%%%%%%%%%%%%

\setlength{\parindent}{0in}
%\setlength{\parskip}{0.0in}
%\setlength{\itemsep}{0in}
%\setlength{\topsep}{0in}
%\setlength{\tabcolsep}{0in}

%%%%%%%%%%%%%%%%%%%%%%%%%%%%%%%%%%%%%%%%%%%%%%%%%%%%%%%%%%%%%%%%%%%%%%%%%%%%%%%%

\newcommand{\class}{Introduction to Cryptography}
\newcommand{\project}{Lab 8 Candidate:\\RSA Parity Oracle}

\renewcommand{\headrulewidth}{0pt}
%\fancyfoot{}
%\cfoot{\quad \quad \quad \quad \quad \quad \thepage\ of \pageref{LastPage}}
% quads to fix center...
\cfoot{}

%%%%%%%%%%%%%%%%%%%%%%%%%%%%%%%%%%%%%%%%%%%%%%%%%%%%%%%%%%%%%%%%%%%%%%%%%%%%%%%%

% Style
\newcommand{\sectionfont}{phv} % Helvetica
\newcommand{\bodyfont}{ppl} % Palatino

\renewcommand{\section}[1] {
    \vspace{12pt}{\quad\fontfamily{\sectionfont}\selectfont\Large\scshape\textbf{#1}}\\[-10pt]
    \vspace{8pt}\rule{\textwidth}{1pt}\\[-16pt]

    % this space is needed
}

\renewcommand{\subsection}[1] {
    \vspace{12pt}{\fontfamily{\sectionfont}\selectfont\large\scshape\textbf{#1}}\\[-10pt]
    %\vspace{8pt}\rule{\textwidth}{1pt}\\[-16pt]

    % this space is needed
}


% \newcommand{\project}[2] {
%                 \begin{tabular}{p{.25\linewidth}p{.05\linewidth}p{.7\linewidth}}
%                         \textbf{#1} & & #2 \\\\
%                 \end{tabular}
% 
%         % this space is needed
% }

%%%%%%%%%%%%%%%%%%%%%%%%%%%%%%%%%%%%%%%%%%%%%%%%%%%%%%%%%%%%%%%%%%%%%%%%%%%%%%%%

\lstset{
language=python,                % choose the language of the code
%basicstyle=\footnotesize,       % the size of the fonts that are used for the code
%numbers=left,                   % where to put the line-numbers
%numberstyle=\footnotesize,      % the size of the fonts that are used for the line-numbers
%stepnumber=2,                   % the step between two line-numbers. If it's 1 each line will be numbered
%%umbersep=5pt,                  % how far the line-numbers are from the code
%backgroundcolor=\color{white},  % choose the background color. You must add \usepackage{color}
showspaces=false,               % show spaces adding particular underscores
showstringspaces=false,         % underline spaces within strings
showtabs=false,                 % show tabs within strings adding particular underscores
%%frame=single,	                % adds a frame around the code
tabsize=4,	                % sets default tabsize to 2 spaces
%%captionpos=b,                   % sets the caption-position to bottom
%%breaklines=true,                % sets automatic line breaking
%%breakatwhitespace=false,        % sets if automatic breaks should only happen at whitespace
%%title=\lstname,                 % show the filename of files included with \lstinputlisting; also try caption instead of title
%escapeinside={\%*}{*)}          % if you want to add a comment within your code
%morekeywords={*,...}            % if you want to add more keywords to the set
}


\begin{document}

\fontfamily{\bodyfont} \selectfont \small
\thispagestyle{empty}
\begin{center}
    \fontfamily{\sectionfont}\selectfont\huge\scshape\textbf{\class}
\end{center}
\begin{center}
    \fontfamily{\sectionfont}\selectfont\large\scshape\textbf{\project}
\end{center}

\section{Overview}

In this lab, you will attack a vulnerability with unpadded / unarmored RSA.
This should feel somewhat similar to a lab you performed earlier in the semester,
when you persuaded a server to encrypt the same message under three different
moduli, and by doing so were able to recover the plaintext.

The oracle for this lab should decrypt a given ciphertext, and return the least
significant bit. Imagine a server that accepted RSA-encrypted messages and
checked the parity of their decryption to validate them, and spat out an error
if they were of the wrong parity.

With enough calls to this oracle, you can decrypt any ciphertext, one bit at a
time.

\section{Assignment}

You will be implementing the majority of the functionality for this lab -- both
server and client. I will give you the base structure that I want you to follow,
but I will ask you to implement the majority. You are of course welcome and
encouraged to use the library that I've given you, as well as code that you
wrote.

\ifinstructornotes
\textbf{Instructor Note:} This is challenge 46 of the Matasano Crypto
Challenges.
\fi

Generate a 1024 bit RSA key pair.

Write an oracle that uses the private key to answer the question ``Is the
plaintext of this message even or odd?'' No other information should be leaked.

Choose a flag. I encourage creativity here, but you must stay under 1024 bits,
or 128 ASCII characters. Encrypt the flag to the public key, creating a
ciphertext.

With the oracle function, you can decrypt the message.

Remember:
\begin{itemize}
	\item RSA ciphertexts are just numbers. You can do trivial math on them. You
		can for instance multiply a ciphertext by the RSA-encryption of another
		number; the corresponding plaintext will be the product of those two
		numbers.
	\item If you double a ciphertext (multiply it by $2^e (\mathrm{mod}) n$, the resulting
		plaintext will be either even or odd.
	\item If the plaintext after doubling is even, doubling the plaintext
		\textit{didn't wrap the modulus} -- the modulus is an odd prime number. That
		means the plaintext is less than half the modulus, which means you have
		roughly one bit of information about the plaintext! Experiment with small
		numbers until you understand this idea.
\end{itemize}

You can repeatedly apply this heuristic, once per bit of the message, checking
your oracle function each time.

Your decryption function starts with bounds for the plaintext of $(0, n]$.

Each iteration of the decryption cuts the bounds in half: either the upper bound
is reduced by half, or the lower bound is.

After log2(n) iterations, you have the decryption of the message.

Print the upper bound of the message as a string at each iteration; you'll see
the message decrypt "hollywood style".

\section{Rules}

Your code should resemble the attacks that I have given you before.
Particularly, you are required to use Pyro4 for remote object calls. You may
distribute the flag already-encrypted, or make it a constant method on a remote
object, as in previous labs.

\section{Grading}
Labs are due Thursday, November 12 at 11:59pm. You will need to turn in
all code, including a complete server, example client, and working solution.

\end{document}
