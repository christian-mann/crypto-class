\newif\ifinstructornotes
%\instructornotestrue % comment out to hide instructor notes

\documentclass[11pt,oneside]{article}
\usepackage{geometry}
\usepackage[T1]{fontenc}
\usepackage{lastpage}
\usepackage{fancyhdr}
\usepackage{hyperref}
\usepackage{graphicx}
\usepackage{listings} % to display code
\usepackage{parskip}
\usepackage{url}
\pagestyle{fancy}
%\geometry{letterpaper,tmargin=.75in,bmargin=1.25in,lmargin=.75in,rmargin=.75in,headheight=13.6pt,headsep=0in,footskip=.3in}
\geometry{letterpaper,tmargin=.5in,bmargin=.5in,lmargin=.5in,rmargin=.5in,headheight=13.6pt,headsep=0in,footskip=.3in}

%%%%%%%%%%%%%%%%%%%%%%%%%%%%%%%%%%%%%%%%%%%%%%%%%%%%%%%%%%%%%%%%%%%%%%%%%%%%%%%%

\setlength{\parindent}{0in}
%\setlength{\parskip}{0.0in}
%\setlength{\itemsep}{0in}
%\setlength{\topsep}{0in}
%\setlength{\tabcolsep}{0in}

%%%%%%%%%%%%%%%%%%%%%%%%%%%%%%%%%%%%%%%%%%%%%%%%%%%%%%%%%%%%%%%%%%%%%%%%%%%%%%%%

\newcommand{\class}{Introduction to Cryptography}
\newcommand{\project}{Lab 8 Candidate:\\CRIME Attack on HTTPS}

\renewcommand{\headrulewidth}{0pt}
%\fancyfoot{}
%\cfoot{\quad \quad \quad \quad \quad \quad \thepage\ of \pageref{LastPage}}
% quads to fix center...
\cfoot{}

%%%%%%%%%%%%%%%%%%%%%%%%%%%%%%%%%%%%%%%%%%%%%%%%%%%%%%%%%%%%%%%%%%%%%%%%%%%%%%%%

% Style
\newcommand{\sectionfont}{phv} % Helvetica
\newcommand{\bodyfont}{ppl} % Palatino

\renewcommand{\section}[1] {
    \vspace{12pt}{\quad\fontfamily{\sectionfont}\selectfont\Large\scshape\textbf{#1}}\\[-10pt]
    \vspace{8pt}\rule{\textwidth}{1pt}\\[-16pt]

    % this space is needed
}

\renewcommand{\subsection}[1] {
    \vspace{12pt}{\fontfamily{\sectionfont}\selectfont\large\scshape\textbf{#1}}\\[-10pt]
    %\vspace{8pt}\rule{\textwidth}{1pt}\\[-16pt]

    % this space is needed
}


% \newcommand{\project}[2] {
%                 \begin{tabular}{p{.25\linewidth}p{.05\linewidth}p{.7\linewidth}}
%                         \textbf{#1} & & #2 \\\\
%                 \end{tabular}
% 
%         % this space is needed
% }

%%%%%%%%%%%%%%%%%%%%%%%%%%%%%%%%%%%%%%%%%%%%%%%%%%%%%%%%%%%%%%%%%%%%%%%%%%%%%%%%

\lstset{
language=python,                % choose the language of the code
%basicstyle=\footnotesize,       % the size of the fonts that are used for the code
%numbers=left,                   % where to put the line-numbers
%numberstyle=\footnotesize,      % the size of the fonts that are used for the line-numbers
%stepnumber=2,                   % the step between two line-numbers. If it's 1 each line will be numbered
%%umbersep=5pt,                  % how far the line-numbers are from the code
%backgroundcolor=\color{white},  % choose the background color. You must add \usepackage{color}
showspaces=false,               % show spaces adding particular underscores
showstringspaces=false,         % underline spaces within strings
showtabs=false,                 % show tabs within strings adding particular underscores
%%frame=single,	                % adds a frame around the code
tabsize=4,	                % sets default tabsize to 2 spaces
%%captionpos=b,                   % sets the caption-position to bottom
%%breaklines=true,                % sets automatic line breaking
%%breakatwhitespace=false,        % sets if automatic breaks should only happen at whitespace
%%title=\lstname,                 % show the filename of files included with \lstinputlisting; also try caption instead of title
%escapeinside={\%*}{*)}          % if you want to add a comment within your code
%morekeywords={*,...}            % if you want to add more keywords to the set
}


\begin{document}

\fontfamily{\bodyfont} \selectfont \small
\thispagestyle{empty}
\begin{center}
    \fontfamily{\sectionfont}\selectfont\huge\scshape\textbf{\class}
\end{center}
\begin{center}
    \fontfamily{\sectionfont}\selectfont\large\scshape\textbf{\project}
\end{center}

\section{Overview}

In this lab, you will re-create the CRIME (``Compression Ratio Information-leak
Made Easy'') attack, which exploited a vulnerability in the interaction between
encryption and compression under certain circumstances. This attack was very
powerful vs HTTPS sessions until very recently.

Essentially, if you are an attacker with:
\begin{enumerate}
	\item Partial plaintext knowledge,
	\item Partial plaintext control,
	\item Access to a compression oracle,
\end{enumerate}

You've got a pretty good chance to recover any additional unknown plaintext.

What's a compression oracle? You give it some input and it tells you how well
the full message compresses, i.e. the length of the resultant output.

This is somewhat similar to a timing attack; we take advantage of an incidental
side channel rather than attacking the crypto mechanisms themselves.

Scenario: You are listening on a network with an eye towards stealing session
cookies. You've injected content allowing you to spawn arbitrary requests, which
you can observe in flight.

\section{Assignment}

You will be implementing the majority of the functionality for this lab -- both
server and client. I will give you the base structure that I want you to follow,
but I will ask you to implement the majority. You are of course welcome and
encouraged to use the library that I've given you, as well as code that you
wrote.

\ifinstructornotes
\textbf{Instructor Note:} This is challenge 46 of the Matasano Crypto
Challenges.
\fi

Come up with a clever flag. The flag should probably be text-based and
(crucially) not terribly repetitive. Encode the flag as base64.

Write an oracle: \texttt{oracle(P) =
length(encrypt(compress(format\_request(P))))}.

Format the request like this:
\begin{verbatim}
POST / HTTP/1.1
Host: gmail.com
Cookie: sessionid=TmV2ZXIgcmV2ZWFsIHRoZSBXdS1UYW5nIFNlY3JldCE=
Content-Length: {len(P)}
{P}
\end{verbatim}

Replace the session ID with your flag.

Compress using zlib.

Encryption is pretty much irrelevant, but do it anyway. Just use some stream
cipher, so that length is preserved. Doesn't matter which. Random key/IV on each
call, whatever.

The idea here is to leak information via the compression algorithm. A payload of
"sessionid=T" should compress a little bit better than "sessionid=S".

There is one issue -- the DEFLATE algorithm operates on bits, but the length
will be in bytes. Even if a better compression is found, it might not cross a
byte boundary. So that's a problem.

Also you might get some false positives. But! You should be able to work around
these issues. Experiment with different strategies, such as passing two copies of the
plaintext, truncating plaintext to various amounts, etc. At each step there
should be one obvious choice for next byte.

Recover the session ID from the compression oracle.

\textbf{Extra:} Swap your stream cipher for CBC and do it again.

\section{Rules}

Your code should resemble the attacks that I have given you before.
Particularly, you are required to use Pyro4 for remote object calls.

\section{Grading}
Labs are due Thursday, November 12 at 11:59pm. You will need to turn in
all code, including a complete server, example client, and working solution.

\end{document}
