\newif\ifinstructornotes
%\instructornotestrue % comment out to hide instructor notes

\documentclass[11pt,oneside]{article}
\usepackage{geometry}
\usepackage[T1]{fontenc}
\usepackage{lastpage}
\usepackage{fancyhdr}
\usepackage{hyperref}
\usepackage{graphicx}
\usepackage{listings} % to display code
\usepackage{multicol}
\usepackage{parskip}
\usepackage{url}
\pagestyle{fancy}
%\geometry{letterpaper,tmargin=.75in,bmargin=1.25in,lmargin=.75in,rmargin=.75in,headheight=13.6pt,headsep=0in,footskip=.3in}
\geometry{letterpaper,tmargin=.5in,bmargin=.5in,lmargin=.5in,rmargin=.5in,headheight=13.6pt,headsep=0in,footskip=.3in}

%%%%%%%%%%%%%%%%%%%%%%%%%%%%%%%%%%%%%%%%%%%%%%%%%%%%%%%%%%%%%%%%%%%%%%%%%%%%%%%%

\setlength{\parindent}{0in}
%\setlength{\parskip}{0.0in}
%\setlength{\itemsep}{0in}
%\setlength{\topsep}{0in}
%\setlength{\tabcolsep}{0in}

%%%%%%%%%%%%%%%%%%%%%%%%%%%%%%%%%%%%%%%%%%%%%%%%%%%%%%%%%%%%%%%%%%%%%%%%%%%%%%%%

\newcommand{\class}{Introduction to Cryptography}
\newcommand{\project}{Lab 8 Candidate:\\Secure Remote Password}

\renewcommand{\headrulewidth}{0pt}
%\fancyfoot{}
%\cfoot{\quad \quad \quad \quad \quad \quad \thepage\ of \pageref{LastPage}}
% quads to fix center...
\cfoot{}

%%%%%%%%%%%%%%%%%%%%%%%%%%%%%%%%%%%%%%%%%%%%%%%%%%%%%%%%%%%%%%%%%%%%%%%%%%%%%%%%

% Style
\newcommand{\sectionfont}{phv} % Helvetica
\newcommand{\bodyfont}{ppl} % Palatino

\renewcommand{\section}[1] {
    \vspace{12pt}{\quad\fontfamily{\sectionfont}\selectfont\Large\scshape\textbf{#1}}\\[-10pt]
    \vspace{8pt}\rule{\textwidth}{1pt}\\[-16pt]

    % this space is needed
}

\renewcommand{\subsection}[1] {
    \vspace{12pt}{\fontfamily{\sectionfont}\selectfont\large\scshape\textbf{#1}}\\[-10pt]
    %\vspace{8pt}\rule{\textwidth}{1pt}\\[-16pt]

    % this space is needed
}


% \newcommand{\project}[2] {
%                 \begin{tabular}{p{.25\linewidth}p{.05\linewidth}p{.7\linewidth}}
%                         \textbf{#1} & & #2 \\\\
%                 \end{tabular}
% 
%         % this space is needed
% }

%%%%%%%%%%%%%%%%%%%%%%%%%%%%%%%%%%%%%%%%%%%%%%%%%%%%%%%%%%%%%%%%%%%%%%%%%%%%%%%%

\lstset{
language=python,                % choose the language of the code
%basicstyle=\footnotesize,       % the size of the fonts that are used for the code
%numbers=left,                   % where to put the line-numbers
%numberstyle=\footnotesize,      % the size of the fonts that are used for the line-numbers
%stepnumber=2,                   % the step between two line-numbers. If it's 1 each line will be numbered
%%umbersep=5pt,                  % how far the line-numbers are from the code
%backgroundcolor=\color{white},  % choose the background color. You must add \usepackage{color}
showspaces=false,               % show spaces adding particular underscores
showstringspaces=false,         % underline spaces within strings
showtabs=false,                 % show tabs within strings adding particular underscores
%%frame=single,	                % adds a frame around the code
tabsize=4,	                % sets default tabsize to 2 spaces
%%captionpos=b,                   % sets the caption-position to bottom
%%breaklines=true,                % sets automatic line breaking
%%breakatwhitespace=false,        % sets if automatic breaks should only happen at whitespace
%%title=\lstname,                 % show the filename of files included with \lstinputlisting; also try caption instead of title
%escapeinside={\%*}{*)}          % if you want to add a comment within your code
%morekeywords={*,...}            % if you want to add more keywords to the set
}


\begin{document}

\fontfamily{\bodyfont} \selectfont \small
\thispagestyle{empty}
\begin{center}
    \fontfamily{\sectionfont}\selectfont\huge\scshape\textbf{\class}
\end{center}
\begin{center}
    \fontfamily{\sectionfont}\selectfont\large\scshape\textbf{\project}
\end{center}

\section{Overview}

In this lab, you will implement Secure Remote Password, and then break it with a
simple attack. While all major implementations of SRP guard against this attack,
many newly created ones do not; it is the first thing you should check if you
need to attack such a system.

Secure Remote Password is similar to Diffie Hellman, but with an extra step of
mixing the password into the public keys. The server also takes an extra step to
avoid storing an easily crackable password-equivalent.

\section{Assignment}

You will be implementing the majority of the functionality for this lab -- both
server and client. I will give you the base structure that I want you to follow,
but I will ask you to implement the majority. You are of course welcome and
encouraged to use the library that I've given you, as well as code that you
wrote.

\ifinstructornotes
\textbf{Instructor Note:} This is challenge 36 of the Matasano Crypto
Challenges.
\fi

\subsection{Implement SRP}
The protocol is as follows. Let C and S be client and server, respectively. All
arithmetic is mod N.

\setlength{\columnseprule}{1pt}
%\begin{multicols}{3}

\textbf{C \& S} agree on
\begin{itemize}
	\item $N$ = [NIST Prime]
	\item $g=2$
	\item $k=3$
	\item $I$ = (email)
	\item $P$ = (password).
\end{itemize}

\textbf{S}
\begin{enumerate}
	\item Generate salt as random integer
	\item Generate string xH = sha256(salt $\|$ password)
	\item Convert xH to integer somehow
	\item Generate $v = g^x$
	\item Save everything but x, xH
\end{enumerate}

\textbf{C $\rightarrow$ S}
\begin{itemize}
	\item $I$
	\item $A = g^a$.
\end{itemize}

\textbf{S $\rightarrow$ C}
\begin{itemize}
	\item salt
	\item $B = kv + g^b$
\end{itemize}

\textbf{S, C compute}
\begin{enumerate}
	\item uH = sha256(A $\|$ B)
	\item u = convert uH to integer
\end{enumerate}

\textbf{C}
\begin{enumerate}
	\item Generate string xH = sha256(salt $\|$ password)
	\item Convert xH to integer somehow
	\item Generate $S = (B - k * g^x)\char`\^(a + u*x)$.
	\item Generate $K$ = sha256(S)
\end{enumerate}

\textbf{S}
\begin{enumerate}
	\item Generate $S = (A * v^u)\char`\^b$.
	\item Generate $K$ = sha256(S).
\end{enumerate}

At this point the key is established, but the client generally sends
HMAC-SHA256(K, salt) to the server to ensure the server key matches the client
key (which ensures that the password is correct).
%\end{multicols}

Trace the equations and convince yourself that if the passwords agree, then the
keys ($K$) will be the same. Then implement this.

\subsection{Part 2: Break SRP with a zero key}

Get SRP working in a client-server setting. Log in with a valid password under
the protocol.

Now log in without your password by having the client send 0 as its $A$ value.
What does this do to the $S$ value that both sides compute?

Now log in without your password by having the client send $N$, $N*2$, etc.

\section{Rules}

Your code should resemble the attacks that I have given you before.
Particularly, you are required to use Pyro4 for remote object calls. You may
distribute the flag already-encrypted, or make it a constant method on a remote
object, as in previous labs.

\section{Grading}
Labs are due Thursday, November 12 at 11:59pm. You will need to turn in
all code, including a complete server, example client, and working solution.

\end{document}
