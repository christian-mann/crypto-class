\newif\ifinstructornotes
%\instructornotestrue % comment out to hide instructor notes

\documentclass[11pt,oneside]{article}
\usepackage{geometry}
\usepackage[T1]{fontenc}
\usepackage{lastpage}
\usepackage{fancyhdr}
\usepackage{hyperref}
\usepackage{graphicx}
\usepackage{listings} % to display code
\usepackage{parskip}
\usepackage{url}
\pagestyle{fancy}
%\geometry{letterpaper,tmargin=.75in,bmargin=1.25in,lmargin=.75in,rmargin=.75in,headheight=13.6pt,headsep=0in,footskip=.3in}
\geometry{letterpaper,tmargin=.5in,bmargin=.5in,lmargin=.5in,rmargin=.5in,headheight=13.6pt,headsep=0in,footskip=.3in}

%%%%%%%%%%%%%%%%%%%%%%%%%%%%%%%%%%%%%%%%%%%%%%%%%%%%%%%%%%%%%%%%%%%%%%%%%%%%%%%%

\setlength{\parindent}{0in}
%\setlength{\parskip}{0.0in}
%\setlength{\itemsep}{0in}
%\setlength{\topsep}{0in}
%\setlength{\tabcolsep}{0in}

%%%%%%%%%%%%%%%%%%%%%%%%%%%%%%%%%%%%%%%%%%%%%%%%%%%%%%%%%%%%%%%%%%%%%%%%%%%%%%%%

\newcommand{\class}{Introduction to Cryptography}
\newcommand{\project}{Lab 9 Candidate:\\Digital Signature Algorithm}

\renewcommand{\headrulewidth}{0pt}
%\fancyfoot{}
%\cfoot{\quad \quad \quad \quad \quad \quad \thepage\ of \pageref{LastPage}}
% quads to fix center...
\cfoot{}

%%%%%%%%%%%%%%%%%%%%%%%%%%%%%%%%%%%%%%%%%%%%%%%%%%%%%%%%%%%%%%%%%%%%%%%%%%%%%%%%

% Style
\newcommand{\sectionfont}{phv} % Helvetica
\newcommand{\bodyfont}{ppl} % Palatino

\renewcommand{\section}[1] {
    \vspace{12pt}{\quad\fontfamily{\sectionfont}\selectfont\Large\scshape\textbf{#1}}\\[-10pt]
    \vspace{8pt}\rule{\textwidth}{1pt}\\[-16pt]

    % this space is needed
}

\renewcommand{\subsection}[1] {
    \vspace{12pt}{\fontfamily{\sectionfont}\selectfont\large\scshape\textbf{#1}}\\[-10pt]
    %\vspace{8pt}\rule{\textwidth}{1pt}\\[-16pt]

    % this space is needed
}


% \newcommand{\project}[2] {
%                 \begin{tabular}{p{.25\linewidth}p{.05\linewidth}p{.7\linewidth}}
%                         \textbf{#1} & & #2 \\\\
%                 \end{tabular}
% 
%         % this space is needed
% }

%%%%%%%%%%%%%%%%%%%%%%%%%%%%%%%%%%%%%%%%%%%%%%%%%%%%%%%%%%%%%%%%%%%%%%%%%%%%%%%%

\lstset{
language=python,                % choose the language of the code
%basicstyle=\footnotesize,       % the size of the fonts that are used for the code
%numbers=left,                   % where to put the line-numbers
%numberstyle=\footnotesize,      % the size of the fonts that are used for the line-numbers
%stepnumber=2,                   % the step between two line-numbers. If it's 1 each line will be numbered
%%umbersep=5pt,                  % how far the line-numbers are from the code
%backgroundcolor=\color{white},  % choose the background color. You must add \usepackage{color}
showspaces=false,               % show spaces adding particular underscores
showstringspaces=false,         % underline spaces within strings
showtabs=false,                 % show tabs within strings adding particular underscores
%%frame=single,	                % adds a frame around the code
tabsize=4,	                % sets default tabsize to 2 spaces
%%captionpos=b,                   % sets the caption-position to bottom
%%breaklines=true,                % sets automatic line breaking
%%breakatwhitespace=false,        % sets if automatic breaks should only happen at whitespace
%%title=\lstname,                 % show the filename of files included with \lstinputlisting; also try caption instead of title
%escapeinside={\%*}{*)}          % if you want to add a comment within your code
%morekeywords={*,...}            % if you want to add more keywords to the set
}


\begin{document}

\fontfamily{\bodyfont} \selectfont \small
\thispagestyle{empty}
\begin{center}
    \fontfamily{\sectionfont}\selectfont\huge\scshape\textbf{\class}
\end{center}
\begin{center}
    \fontfamily{\sectionfont}\selectfont\large\scshape\textbf{\project}
\end{center}

\section{Overview}

This lab contains two relatively straightforward attacks on the Digital
Signature Algorithm.

You will also implement DSA yourself, including signing and verification, but
not including parameter generation.

The first section demonstrates that DSA private keys(!) can be recovered from a
signed message and a nonce.

The second section demonstrates that DSA nonces can be recovered from a pair of
messages signed with the same nonce.


\section{Assignment}

You will be implementing the majority of the functionality for this lab -- both
server and client. I will give you the base structure that I want you to follow,
but I will ask you to implement the majority. You are of course welcome and
encouraged to use the library that I've given you, as well as code that you
wrote.

\ifinstructornotes
\textbf{Instructor Note:} This is challenges 43-45 of the Matasano Crypto
Challenges.
\fi

\subsection{Part 0: Implement DSA}

This is pretty straightforward. There is a set of equations online that you need
to follow. You do not need to implement parameter generation; you can use these:
\begin{verbatim}
p = 800000000000000089e1855218a0e7dac38136ffafa72eda7
    859f2171e25e65eac698c1702578b07dc2a1076da241c76c6
    2d374d8389ea5aeffd3226a0530cc565f3bf6b50929139ebe
    ac04f48c3c84afb796d61e5a4f9a8fda812ab59494232c7d2
    b4deb50aa18ee9e132bfa85ac4374d7f9091abc3d015efc87
    1a584471bb1

q = f4f47f05794b256174bba6e9b396a7707e563c5b

g = 5958c9d3898b224b12672c0b98e06c60df923cb8bc999d119
    458fef538b8fa4046c8db53039db620c094c9fa077ef389b5
    322a559946a71903f990f1f7e0e025e2d7f7cf494aff1a047
    0f5b64c36b625a097f1651fe775323556fe00b3608c887892
    878480e99041be601a62166ca6894bdd41a7054ec89f756ba
    9fc95302291
\end{verbatim}

\subsection{Part 1: Key Recovery From Nonce}

The DSA signing operation generates a random subkey $k$. You know this because
you implemented the DSA sign operation.

Given a known key $k$, it's trivial to recover the DSA private key $x$:
\[ x = \frac{s*k - H(\mathrm{msg})}{r} \mathrm mod q \]

Do this a couple times to prove to yourself that you grok it. Capture it in a
function of some sort.

Use your parameters. Generate a keypair; publish your public key.

Sign a message. I'm not sure what NIST wants you to do with the hash to convert
it to an integer, but just stick ``0x'' in front of it and parse it as hex.

However, when you sign the message, use a broken implementation of DSA that
chooses $k$ values between $0$ and $2^{16}$. Recover the public key.

\subsection{Part 2: DSA Nonce Recovery from Repeated Nonce}

I will supply a collection of DSA-signed messages\footnote{see appendix}. They
were signed under the following public key:
\begin{verbatim}
y = 2d026f4bf30195ede3a088da85e398ef869611d0f68f07
    13d51c9c1a3a26c95105d915e2d8cdf26d056b86b8a7b8
    5519b1c23cc3ecdc6062650462e3063bd179c2a6581519
    f674a61f1d89a1fff27171ebc1b93d4dc57bceb7ae2430
    f98a6a4d83d8279ee65d71c1203d2c96d65ebbf7cce9d3
    2971c3de5084cce04a2e147821
\end{verbatim}

It should not be hard to find the messages for which I have accidentally used a
repeated $k$ value (hint: what variable in the signature directly depends on
$k$?). Given a pair of such messages, you can discover the $k$ used with the
following formula:

\[ k = \frac{m_1 - m_2}{s_1 - s_2} \mathrm{mod} q \]

You should work out this formula from the original DSA equations to demystify
it. Remember that group arithmetic wants to mess with you. Everything is mod
$q$, which means that \textit{every} value should be s.t. $0 \leq x < q$; not
negative, not greater than $q$. Also remember that division is multiplication by
modular inverse.

What's my private key?

Note: This attack (in an elliptic curve group) broke the PS3.

\section{Grading}

Labs are due Thursday, November 12 at 11:59pm. You will need to turn in
all code, including a complete solution to each of the three problems above, and
my private key from part 2.

\appendix
\section{Encrypted Messages for Part 2}
\begin{verbatim}
msg: Listen for me, you better listen for me now. 
s: 1267396447369736888040262262183731677867615804316
r: 1105520928110492191417703162650245113664610474875
m: a4db3de27e2db3e5ef085ced2bced91b82e0df19
msg: Listen for me, you better listen for me now. 
s: 29097472083055673620219739525237952924429516683
r: 51241962016175933742870323080382366896234169532
m: a4db3de27e2db3e5ef085ced2bced91b82e0df19
msg: When me rockin' the microphone me rock on steady, 
s: 277954141006005142760672187124679727147013405915
r: 228998983350752111397582948403934722619745721541
m: 21194f72fe39a80c9c20689b8cf6ce9b0e7e52d4
msg: Yes a Daddy me Snow me are de article dan. 
s: 1013310051748123261520038320957902085950122277350
r: 1099349585689717635654222811555852075108857446485
m: 1d7aaaa05d2dee2f7dabdc6fa70b6ddab9c051c5
msg: But in a in an' a out de dance em 
s: 203941148183364719753516612269608665183595279549
r: 425320991325990345751346113277224109611205133736
m: 6bc188db6e9e6c7d796f7fdd7fa411776d7a9ff
msg: Aye say where you come from a, 
s: 502033987625712840101435170279955665681605114553
r: 486260321619055468276539425880393574698069264007
m: 5ff4d4e8be2f8aae8a5bfaabf7408bd7628f43c9
msg: People em say ya come from Jamaica, 
s: 1133410958677785175751131958546453870649059955513
r: 537050122560927032962561247064393639163940220795
m: 7d9abd18bbecdaa93650ecc4da1b9fcae911412
msg: But me born an' raised in the ghetto that I want yas to know, 
s: 559339368782867010304266546527989050544914568162
r: 826843595826780327326695197394862356805575316699
m: 88b9e184393408b133efef59fcef85576d69e249
msg: Pure black people mon is all I mon know. 
s: 1021643638653719618255840562522049391608552714967
r: 1105520928110492191417703162650245113664610474875
m: d22804c4899b522b23eda34d2137cd8cc22b9ce8
msg: Yeah me shoes a an tear up an' now me toes is a show a 
s: 506591325247687166499867321330657300306462367256
r: 51241962016175933742870323080382366896234169532
m: bc7ec371d951977cba10381da08fe934dea80314
msg: Where me a born in are de one Toronto, so 
s: 458429062067186207052865988429747640462282138703
r: 228998983350752111397582948403934722619745721541
m: d6340bfcda59b6b75b59ca634813d572de800e8f
\end{verbatim}

\end{document}
