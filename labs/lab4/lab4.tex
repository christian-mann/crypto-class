\newif\ifinstructornotes
%\instructornotestrue % comment out to hide instructor notes

\documentclass[11pt,oneside]{article}
\usepackage{geometry}
\usepackage[T1]{fontenc}
\usepackage{lastpage}
\usepackage{fancyhdr}
\usepackage{hyperref}
\usepackage{graphicx}
\usepackage{listings} % to display code
\usepackage{parskip}
\usepackage{url}
\pagestyle{fancy}
%\geometry{letterpaper,tmargin=.75in,bmargin=1.25in,lmargin=.75in,rmargin=.75in,headheight=13.6pt,headsep=0in,footskip=.3in}
\geometry{letterpaper,tmargin=.5in,bmargin=.5in,lmargin=.5in,rmargin=.5in,headheight=13.6pt,headsep=0in,footskip=.3in}

%%%%%%%%%%%%%%%%%%%%%%%%%%%%%%%%%%%%%%%%%%%%%%%%%%%%%%%%%%%%%%%%%%%%%%%%%%%%%%%%

\setlength{\parindent}{0in}
%\setlength{\parskip}{0.0in}
%\setlength{\itemsep}{0in}
%\setlength{\topsep}{0in}
%\setlength{\tabcolsep}{0in}

%%%%%%%%%%%%%%%%%%%%%%%%%%%%%%%%%%%%%%%%%%%%%%%%%%%%%%%%%%%%%%%%%%%%%%%%%%%%%%%%

\newcommand{\class}{Introduction to Cryptography}
\newcommand{\project}{Lab 4 Assignment:\\RSA Broadcast Attack}

\renewcommand{\headrulewidth}{0pt}
%\fancyfoot{}
%\cfoot{\quad \quad \quad \quad \quad \quad \thepage\ of \pageref{LastPage}}
% quads to fix center...
\cfoot{}

%%%%%%%%%%%%%%%%%%%%%%%%%%%%%%%%%%%%%%%%%%%%%%%%%%%%%%%%%%%%%%%%%%%%%%%%%%%%%%%%

% Style
\newcommand{\sectionfont}{phv} % Helvetica
\newcommand{\bodyfont}{ppl} % Palatino

\renewcommand{\section}[1] {
    \vspace{12pt}{\quad\fontfamily{\sectionfont}\selectfont\Large\scshape\textbf{#1}}\\[-10pt]
    \vspace{8pt}\rule{\textwidth}{1pt}\\[-16pt]

    % this space is needed
}

\renewcommand{\subsection}[1] {
    \vspace{12pt}{\fontfamily{\sectionfont}\selectfont\large\scshape\textbf{#1}}\\[-10pt]
    %\vspace{8pt}\rule{\textwidth}{1pt}\\[-16pt]

    % this space is needed
}


% \newcommand{\project}[2] {
%                 \begin{tabular}{p{.25\linewidth}p{.05\linewidth}p{.7\linewidth}}
%                         \textbf{#1} & & #2 \\\\
%                 \end{tabular}
% 
%         % this space is needed
% }

%%%%%%%%%%%%%%%%%%%%%%%%%%%%%%%%%%%%%%%%%%%%%%%%%%%%%%%%%%%%%%%%%%%%%%%%%%%%%%%%

\lstset{
language=python,                % choose the language of the code
%basicstyle=\footnotesize,       % the size of the fonts that are used for the code
%numbers=left,                   % where to put the line-numbers
%numberstyle=\footnotesize,      % the size of the fonts that are used for the line-numbers
%stepnumber=2,                   % the step between two line-numbers. If it's 1 each line will be numbered
%%umbersep=5pt,                  % how far the line-numbers are from the code
%backgroundcolor=\color{white},  % choose the background color. You must add \usepackage{color}
showspaces=false,               % show spaces adding particular underscores
showstringspaces=false,         % underline spaces within strings
showtabs=false,                 % show tabs within strings adding particular underscores
%%frame=single,	                % adds a frame around the code
tabsize=4,	                % sets default tabsize to 2 spaces
%%captionpos=b,                   % sets the caption-position to bottom
%%breaklines=true,                % sets automatic line breaking
%%breakatwhitespace=false,        % sets if automatic breaks should only happen at whitespace
%%title=\lstname,                 % show the filename of files included with \lstinputlisting; also try caption instead of title
%escapeinside={\%*}{*)}          % if you want to add a comment within your code
%morekeywords={*,...}            % if you want to add more keywords to the set
}


\begin{document}

\fontfamily{\bodyfont} \selectfont \small
\thispagestyle{empty}
\begin{center}
    \fontfamily{\sectionfont}\selectfont\huge\scshape\textbf{\class}
\end{center}
\begin{center}
    \fontfamily{\sectionfont}\selectfont\large\scshape\textbf{\project}
\end{center}

\section{Overview}

This short lab will demonstrate the fragility of unpadded RSA. You will recover
a message that has been encrypted multiple times, under different keys, via
H\aa{}stad's Broadcast attack.

There are two sections of this lab. The second section is very similar to the
first, and will be solved with almost the same code.

The services are available as Pyro4\footnote{https://pythonhosted.org/Pyro4/}
remote objects. This should work with both Python 2 and 3; let me know if
technical issues get in the way and we can work something out.

\section{Assignment}

\subsection{Part 1: E = 3 RSA Broadcast}

\textit{Explanatory text largely taken from Matasano crypto challenge 40.}

Assume you're a Javascript programmer. That is, you're using a naive handrolled
RSA to encrypt without padding.

Assume you can be convinced or coerced into encrypting the same plaintext three
times, under three different public keys, with E = 3. You can; it's happened.

Then an attacker can trivially decrypt your message, by: \begin{enumerate}
	\item Capturing any 3 of the ciphertexts and their corresponding pubkeys.
	\item Using the Chinese Remainder Theorem to solve for the number
		represented by the three ciphertexts (which are residues modulo their
		respective pubkeys).
	\item Taking the cube root of the resulting number.
\end{enumerate}

The Chinese Remainder Theorem says you can take any number and represent it as
the combination of a series of residues mod a series of moduli. In the
three-residue case, you have:
\begin{lstlisting}[language={}]
	result = 
		(c_0 * m_s_0 * invmod(m_s_0, n_0)) + 
		(c_1 * m_s_1 * invmod(m_s_1, n_1)) + 
		(c_2 * m_s_2 * invmod(m_s_2, n_2)) mod N_012
\end{lstlisting}

where

\begin{lstlisting}[language={}]
	c_0, c_1, c_2 are the three respective residues mod n_0, n_1, n_2

	m_s_n (for n in 0, 1, 2) are the product of the moduli EXCEPT n_n -- ie,
	m_s_1 is n_0 * n_2

	N_012 is the product of all the moduli
\end{lstlisting}

To decrypt RSA using a simple cube root, take the accumulated result and
cube-root it.

\begin{verbatim}
Server URL: PYRO:RSABroadcast@10.10.200.42:9041
\end{verbatim}

\ifinstructornotes
\textbf{Instructor Note:} This is challenge 40 of the Matasano crypto
challenges.
\fi

\subsection{Part 2: E $\neq$ 3 Broadcast Attack}

Imagine you, the same naive Javascript programmer, have discovered that your
handrolled RSA encryption can be defeated by collecting enough ciphertexts where
E = 3. You have thus decided to disallow all keys with E = 3, and only
generate keys where E $\geq$ 5.

This is still vulnerable to the same attack. You will need to change some
constants in your code, and \textit{should} abstract some logic into loops. A
perfect solution will be able to attack the system regardless of which E values
are disallowed.

\begin{verbatim}
Server URL: PYRO:RSABroadcastNo3@10.10.200.42:9041
\end{verbatim}

\ifinstructornotes
\textbf{Instructor Note:} This is an original challenge! But it closely
resembles challege 40.
\fi

\section{Mathematics}

The ``Hast\aa{}d's Broadcast Attack'' section of the ``Coppersmith's Attack''
Wikipedia page describes the attack from a mathematical perspective; I would
recommend demonstrating the attack with a very small key, on your own.

\section{Rules}

Please do not exploit any non-cryptographic vulnerabilities you may discover on
any machines. You should not need to attempt\footnote{I hate telling people they
are not allowed to hack my services. If you want to try, please go ahead. If you
find something, just tell me, and don't use it to cheat on the lab.} to gain any
level of access to the machine beyond a TCP connection, nor should you attempt
to mount a Denial of Service attack on any host on the network.

\section{Grading}
Labs are due Thursday, October 15 at 11:59pm. You will need to turn in all
recovered flags and any supporting code.

\end{document}
