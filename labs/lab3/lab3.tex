\newif\ifinstructornotes
%\instructornotestrue % comment out to hide instructor notes

\documentclass[11pt,oneside]{article}
\usepackage{geometry}
\usepackage[T1]{fontenc}
\usepackage{lastpage}
\usepackage{fancyhdr}
\usepackage{hyperref}
\usepackage{graphicx}
\usepackage{listings} % to display code
\usepackage{parskip}
\usepackage{url}
\pagestyle{fancy}
%\geometry{letterpaper,tmargin=.75in,bmargin=1.25in,lmargin=.75in,rmargin=.75in,headheight=13.6pt,headsep=0in,footskip=.3in}
\geometry{letterpaper,tmargin=.5in,bmargin=.5in,lmargin=.5in,rmargin=.5in,headheight=13.6pt,headsep=0in,footskip=.3in}

%%%%%%%%%%%%%%%%%%%%%%%%%%%%%%%%%%%%%%%%%%%%%%%%%%%%%%%%%%%%%%%%%%%%%%%%%%%%%%%%

\setlength{\parindent}{0in}
%\setlength{\parskip}{0.0in}
%\setlength{\itemsep}{0in}
%\setlength{\topsep}{0in}
%\setlength{\tabcolsep}{0in}

%%%%%%%%%%%%%%%%%%%%%%%%%%%%%%%%%%%%%%%%%%%%%%%%%%%%%%%%%%%%%%%%%%%%%%%%%%%%%%%%

\newcommand{\class}{Introduction to Cryptography}
\newcommand{\project}{Lab 3 Assignment:\\Block Chaining Attacks}

\renewcommand{\headrulewidth}{0pt}
%\fancyfoot{}
%\cfoot{\quad \quad \quad \quad \quad \quad \thepage\ of \pageref{LastPage}}
% quads to fix center...
\cfoot{}

%%%%%%%%%%%%%%%%%%%%%%%%%%%%%%%%%%%%%%%%%%%%%%%%%%%%%%%%%%%%%%%%%%%%%%%%%%%%%%%%

% Style
\newcommand{\sectionfont}{phv} % Helvetica
\newcommand{\bodyfont}{ppl} % Palatino

\renewcommand{\section}[1] {
    \vspace{12pt}{\quad\fontfamily{\sectionfont}\selectfont\Large\scshape\textbf{#1}}\\[-10pt]
    \vspace{8pt}\rule{\textwidth}{1pt}\\[-16pt]

    % this space is needed
}

\renewcommand{\subsection}[1] {
    \vspace{12pt}{\fontfamily{\sectionfont}\selectfont\large\scshape\textbf{#1}}\\[-10pt]
    %\vspace{8pt}\rule{\textwidth}{1pt}\\[-16pt]

    % this space is needed
}


% \newcommand{\project}[2] {
%                 \begin{tabular}{p{.25\linewidth}p{.05\linewidth}p{.7\linewidth}}
%                         \textbf{#1} & & #2 \\\\
%                 \end{tabular}
% 
%         % this space is needed
% }

%%%%%%%%%%%%%%%%%%%%%%%%%%%%%%%%%%%%%%%%%%%%%%%%%%%%%%%%%%%%%%%%%%%%%%%%%%%%%%%%

\lstset{
language=python,                % choose the language of the code
%basicstyle=\footnotesize,       % the size of the fonts that are used for the code
%numbers=left,                   % where to put the line-numbers
%numberstyle=\footnotesize,      % the size of the fonts that are used for the line-numbers
%stepnumber=2,                   % the step between two line-numbers. If it's 1 each line will be numbered
%%umbersep=5pt,                  % how far the line-numbers are from the code
%backgroundcolor=\color{white},  % choose the background color. You must add \usepackage{color}
showspaces=false,               % show spaces adding particular underscores
showstringspaces=false,         % underline spaces within strings
showtabs=false,                 % show tabs within strings adding particular underscores
%%frame=single,	                % adds a frame around the code
tabsize=4,	                % sets default tabsize to 2 spaces
%%captionpos=b,                   % sets the caption-position to bottom
%%breaklines=true,                % sets automatic line breaking
%%breakatwhitespace=false,        % sets if automatic breaks should only happen at whitespace
%%title=\lstname,                 % show the filename of files included with \lstinputlisting; also try caption instead of title
%escapeinside={\%*}{*)}          % if you want to add a comment within your code
%morekeywords={*,...}            % if you want to add more keywords to the set
}


\begin{document}

\fontfamily{\bodyfont} \selectfont \small
\thispagestyle{empty}
\begin{center}
    \fontfamily{\sectionfont}\selectfont\huge\scshape\textbf{\class}
\end{center}
\begin{center}
    \fontfamily{\sectionfont}\selectfont\large\scshape\textbf{\project}
\end{center}

\section{Overview}

This lab will involve various attacks on block ciphers. While the cipher used
will be AES, this should not affect your decryption code in any way. Successful
attacks rarely target the core cipher; rather, you will be attacking the
\textit{mode} of encryption used. Particularly, for this project you will be
mounting attacks on the CBC mode of encryption.

Each of these challenges will involve a service on the network, giving out
ciphertexts and returning responses. Your code will need to interact with this
service. In every case, the end goal will be the retrieval of a flag. This will
be accomplished in different ways, depending on the challenge.

Learning to program using sockets is not the goal of this class; you can take Network
Security or CNO for that. Thus, I will provide example socket code for any
languages I know are being used. If you have a language that does not support
network activity, I will be excited to work with you to arrive at a solution.

\section{Assignment}

\subsection{Part 1: CBC Bitflip Attacks}

This section is similar to the ECB Forge section of Lab 2. A service is set up
that will take some amount of user data, prepend the string:
\texttt{comment1=cooking\%20MCs;userdata=}, and append the string:
\texttt{;comment2=\%20like\%20a\%20pound\%20of\%20bacon}.

The service removes the ';' and '=' characters.

The function then pads the input to the 16-byte AES block length, and encrypts
it with a static unknown AES key.

The other end of the service decrypts the string and looks for
\texttt{;admin=true;}.

Modify the ciphertext to cause the second function to return true.

Use the fact that in CBC mode, a 1-bit error in a ciphertext block:
\begin{itemize}
	\item Completely scrambles the block the error occurs in
	\item Produces the identical 1-bit error(/edit) in the next ciphertext block.
\end{itemize}

\begin{verbatim}
Server URL: http://10.10.200.42:5031
Bake Cookie: /update_userdata?userdata=DEADBEEF0123
Taste Cookie: /get_flag?cookie=ABCDEF&iv=ABCDEF
\end{verbatim}

\ifinstructornotes
\textbf{Instructor Note:} This is challenge 16 of the Matasano crypto
challenges.
\fi

\subsection{Part 2: Padding Oracle Attack}

In this section, you will decrypt data that is being secured with AES-CBC under
an adaptive chosen plaintext attack.

The first URL, a generator, will encrypt the flag under an unknown key and a
random, known IV.

The second URL will consume a ciphertext/IV and determine whether the plaintext
is padded correctly. It will return this to the user.

Use these two oracles to determine the flag.

There are many, many writeups available that explain how to exploit a padding
oracle attack, but my favorite is
\url{http://robertheaton.com/2013/07/29/padding-oracle-attack/}

\begin{verbatim}
Server URL: http://10.10.200.42:5032
Generate Ciphertext: /get_ciphertext
Check Padding: /check_padding?encrypted_data=ABCDEF&iv=ABCDEF
\end{verbatim}

\ifinstructornotes
\textbf{Instructor Note:} This is challenge 17 of the Matasano crypto
challenges.
\fi

\section{Rules}

Please do not exploit any non-cryptographic vulnerabilities you may discover on
any machines. You should not need to attempt\footnote{I hate telling people they
are not allowed to hack my services. If you want to try, please go ahead. If you
find something, just tell me, and don't use it to cheat on the lab.} to gain any
level of access to the machine beyond a TCP connection, nor should you attempt
to mount a Denial of Service attack on any host on the network.

\section{Grading}
Labs are due Tuesday, September 29 at 11:59pm. You will need to turn in all
recovered flags and any supporting code.

\end{document}
