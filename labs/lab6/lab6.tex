\newif\ifinstructornotes
%\instructornotestrue % comment out to hide instructor notes

\documentclass[11pt,oneside]{article}
\usepackage{geometry}
\usepackage[T1]{fontenc}
\usepackage{lastpage}
\usepackage{fancyhdr}
\usepackage{hyperref}
\usepackage{graphicx}
\usepackage{listings} % to display code
\usepackage{parskip}
\usepackage{url}
\pagestyle{fancy}
%\geometry{letterpaper,tmargin=.75in,bmargin=1.25in,lmargin=.75in,rmargin=.75in,headheight=13.6pt,headsep=0in,footskip=.3in}
\geometry{letterpaper,tmargin=.5in,bmargin=.5in,lmargin=.5in,rmargin=.5in,headheight=13.6pt,headsep=0in,footskip=.3in}

%%%%%%%%%%%%%%%%%%%%%%%%%%%%%%%%%%%%%%%%%%%%%%%%%%%%%%%%%%%%%%%%%%%%%%%%%%%%%%%%

\setlength{\parindent}{0in}
%\setlength{\parskip}{0.0in}
%\setlength{\itemsep}{0in}
%\setlength{\topsep}{0in}
%\setlength{\tabcolsep}{0in}

%%%%%%%%%%%%%%%%%%%%%%%%%%%%%%%%%%%%%%%%%%%%%%%%%%%%%%%%%%%%%%%%%%%%%%%%%%%%%%%%

\newcommand{\class}{Introduction to Cryptography}
\newcommand{\project}{Lab 6 Assignment:\\RSA Insecure Prime Generation}

\renewcommand{\headrulewidth}{0pt}
%\fancyfoot{}
%\cfoot{\quad \quad \quad \quad \quad \quad \thepage\ of \pageref{LastPage}}
% quads to fix center...
\cfoot{}

%%%%%%%%%%%%%%%%%%%%%%%%%%%%%%%%%%%%%%%%%%%%%%%%%%%%%%%%%%%%%%%%%%%%%%%%%%%%%%%%

% Style
\newcommand{\sectionfont}{phv} % Helvetica
\newcommand{\bodyfont}{ppl} % Palatino

\renewcommand{\section}[1] {
    \vspace{12pt}{\quad\fontfamily{\sectionfont}\selectfont\Large\scshape\textbf{#1}}\\[-10pt]
    \vspace{8pt}\rule{\textwidth}{1pt}\\[-16pt]

    % this space is needed
}

\renewcommand{\subsection}[1] {
    \vspace{12pt}{\fontfamily{\sectionfont}\selectfont\large\scshape\textbf{#1}}\\[-10pt]
    %\vspace{8pt}\rule{\textwidth}{1pt}\\[-16pt]

    % this space is needed
}


% \newcommand{\project}[2] {
%                 \begin{tabular}{p{.25\linewidth}p{.05\linewidth}p{.7\linewidth}}
%                         \textbf{#1} & & #2 \\\\
%                 \end{tabular}
% 
%         % this space is needed
% }

%%%%%%%%%%%%%%%%%%%%%%%%%%%%%%%%%%%%%%%%%%%%%%%%%%%%%%%%%%%%%%%%%%%%%%%%%%%%%%%%

\lstset{
language=python,                % choose the language of the code
%basicstyle=\footnotesize,       % the size of the fonts that are used for the code
%numbers=left,                   % where to put the line-numbers
%numberstyle=\footnotesize,      % the size of the fonts that are used for the line-numbers
%stepnumber=2,                   % the step between two line-numbers. If it's 1 each line will be numbered
%%umbersep=5pt,                  % how far the line-numbers are from the code
%backgroundcolor=\color{white},  % choose the background color. You must add \usepackage{color}
showspaces=false,               % show spaces adding particular underscores
showstringspaces=false,         % underline spaces within strings
showtabs=false,                 % show tabs within strings adding particular underscores
%%frame=single,	                % adds a frame around the code
tabsize=4,	                % sets default tabsize to 2 spaces
%%captionpos=b,                   % sets the caption-position to bottom
%%breaklines=true,                % sets automatic line breaking
%%breakatwhitespace=false,        % sets if automatic breaks should only happen at whitespace
%%title=\lstname,                 % show the filename of files included with \lstinputlisting; also try caption instead of title
%escapeinside={\%*}{*)}          % if you want to add a comment within your code
%morekeywords={*,...}            % if you want to add more keywords to the set
}


\begin{document}

\fontfamily{\bodyfont} \selectfont \small
\thispagestyle{empty}
\begin{center}
    \fontfamily{\sectionfont}\selectfont\huge\scshape\textbf{\class}
\end{center}
\begin{center}
    \fontfamily{\sectionfont}\selectfont\large\scshape\textbf{\project}
\end{center}

\section{Overview}

In this lab, you will discover why a secure random number generator is so
crucial for RSA key generation.

This lab involves an RSA keypair generator. You will
have the ability to generate any number of key pairs, but will only be given the
public key. This, by itself, is not insecure.  \textit{However}, the service
selects primes in a very insecure manner. You can use this vulnerability to
recover private keys, and decrypt a sample message.

\section{Assignment}

You have intercepted a message that has been encrypted under an RSA public key.
The exponent of the key is large enough that simply taking the cube root, fifth
root, etc. will not work, as in the last challenge.

You have access to the key generation server that was used to generate the
public key in question. You are free to generate as many keys as you would like,
although as you will not have access to the respective private keys this will
not immediately allow you to decrypt the message.

In order to save energy, the service does not generate two new primes for each
key; instead, it has a static set of primes, from which it chooses two. You do
not have access to this set.

Use the oracle to factor the given public modulus, reconstruct the private key,
and decrypt the message. Use the fact that the gcd (greatest common divisor)
function is very quick to compute.

\textbf{Note:} This is a somewhat simplified version of a very legitimate
attack. If two independent programs use the same RNG seed to generate primes,
they are liable to generate the same primes. This weakens both keys. In 2012,
Arjen Lenstra and colleagues published ``Ron Was Wrong, Whit Was Right,'' which,
along with loads of questionable opinions about RSA vs ElGamal, contains the
results of a survey performed against millions of RSA moduli. They found that
around 0.2\% of them shared a prime factor, and were thus easily factorable.

\ifinstructornotes
\textbf{Instructor Note:} This comes loosely from a PicoCTF 2014 challenge.
\fi

\section{Rules}

Standard rules apply. Please do not exploit any non-cryptographic
vulnerabilities you may discover on any machines. You should not need to attempt
to gain any level of access to the machine beyond a TCP connection, nor should
you attempt to mount a Denial of Service attack on any host on the network.

\section{Grading}
Labs are due Thursday, November 5 at 11:59pm. You will need to turn in
all relevant code, and the recovered flag.

\end{document}
