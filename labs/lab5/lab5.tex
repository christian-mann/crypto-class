\newif\ifinstructornotes
%\instructornotestrue % comment out to hide instructor notes

\documentclass[11pt,oneside]{article}
\usepackage{geometry}
\usepackage[T1]{fontenc}
\usepackage{lastpage}
\usepackage{fancyhdr}
\usepackage{hyperref}
\usepackage{graphicx}
\usepackage{listings} % to display code
\usepackage{parskip}
\usepackage{url}
\pagestyle{fancy}
%\geometry{letterpaper,tmargin=.75in,bmargin=1.25in,lmargin=.75in,rmargin=.75in,headheight=13.6pt,headsep=0in,footskip=.3in}
\geometry{letterpaper,tmargin=.5in,bmargin=.5in,lmargin=.5in,rmargin=.5in,headheight=13.6pt,headsep=0in,footskip=.3in}

%%%%%%%%%%%%%%%%%%%%%%%%%%%%%%%%%%%%%%%%%%%%%%%%%%%%%%%%%%%%%%%%%%%%%%%%%%%%%%%%

\setlength{\parindent}{0in}
%\setlength{\parskip}{0.0in}
%\setlength{\itemsep}{0in}
%\setlength{\topsep}{0in}
%\setlength{\tabcolsep}{0in}

%%%%%%%%%%%%%%%%%%%%%%%%%%%%%%%%%%%%%%%%%%%%%%%%%%%%%%%%%%%%%%%%%%%%%%%%%%%%%%%%

\newcommand{\class}{Introduction to Cryptography}
\newcommand{\project}{Lab 5 Assignment:\\Hash-Length Extension}

\renewcommand{\headrulewidth}{0pt}
%\fancyfoot{}
%\cfoot{\quad \quad \quad \quad \quad \quad \thepage\ of \pageref{LastPage}}
% quads to fix center...
\cfoot{}

%%%%%%%%%%%%%%%%%%%%%%%%%%%%%%%%%%%%%%%%%%%%%%%%%%%%%%%%%%%%%%%%%%%%%%%%%%%%%%%%

% Style
\newcommand{\sectionfont}{phv} % Helvetica
\newcommand{\bodyfont}{ppl} % Palatino

\renewcommand{\section}[1] {
    \vspace{12pt}{\quad\fontfamily{\sectionfont}\selectfont\Large\scshape\textbf{#1}}\\[-10pt]
    \vspace{8pt}\rule{\textwidth}{1pt}\\[-16pt]

    % this space is needed
}

\renewcommand{\subsection}[1] {
    \vspace{12pt}{\fontfamily{\sectionfont}\selectfont\large\scshape\textbf{#1}}\\[-10pt]
    %\vspace{8pt}\rule{\textwidth}{1pt}\\[-16pt]

    % this space is needed
}


% \newcommand{\project}[2] {
%                 \begin{tabular}{p{.25\linewidth}p{.05\linewidth}p{.7\linewidth}}
%                         \textbf{#1} & & #2 \\\\
%                 \end{tabular}
% 
%         % this space is needed
% }

%%%%%%%%%%%%%%%%%%%%%%%%%%%%%%%%%%%%%%%%%%%%%%%%%%%%%%%%%%%%%%%%%%%%%%%%%%%%%%%%

\lstset{
language=python,                % choose the language of the code
%basicstyle=\footnotesize,       % the size of the fonts that are used for the code
%numbers=left,                   % where to put the line-numbers
%numberstyle=\footnotesize,      % the size of the fonts that are used for the line-numbers
%stepnumber=2,                   % the step between two line-numbers. If it's 1 each line will be numbered
%%umbersep=5pt,                  % how far the line-numbers are from the code
%backgroundcolor=\color{white},  % choose the background color. You must add \usepackage{color}
showspaces=false,               % show spaces adding particular underscores
showstringspaces=false,         % underline spaces within strings
showtabs=false,                 % show tabs within strings adding particular underscores
%%frame=single,	                % adds a frame around the code
tabsize=4,	                % sets default tabsize to 2 spaces
%%captionpos=b,                   % sets the caption-position to bottom
%%breaklines=true,                % sets automatic line breaking
%%breakatwhitespace=false,        % sets if automatic breaks should only happen at whitespace
%%title=\lstname,                 % show the filename of files included with \lstinputlisting; also try caption instead of title
%escapeinside={\%*}{*)}          % if you want to add a comment within your code
%morekeywords={*,...}            % if you want to add more keywords to the set
}


\begin{document}

\fontfamily{\bodyfont} \selectfont \small
\thispagestyle{empty}
\begin{center}
    \fontfamily{\sectionfont}\selectfont\huge\scshape\textbf{\class}
\end{center}
\begin{center}
    \fontfamily{\sectionfont}\selectfont\large\scshape\textbf{\project}
\end{center}

\section{Overview}

In this lab, you will implement and then break an insecure Message
Authentication Code scheme. Particularly, the MAC under discussion is
secret-prefix SHA-1, which is trivially breakable by a length extension
attack\footnote{As are MD4, MD5, SHA-2, and anything else that uses the
Merkle-Damgard construction}.

There is (shockingly) only one section to this lab, but it will require you to
write a fairly large amount of code on your own. While you can (and may) obtain an
implementation of SHA-1 online, you will need to modify it somewhat to allow
injection of an arbitrary starting point.

There is no web service associated with this lab. While I could set one up, I
feel the lab speaks for itself, and there is no natural way to hide a key to be
decrypted later, in this case.

\section{Assignment}

\textit{Just kidding, there totally are two parts. But they're pretty closely
related; I almost put them in one part.}

\subsection{Part 1: Implement SHA-1 secret-prefix MAC}

Find a SHA-1 implementation in your favorite language. Note: you will need to
obtain an implementation \textit{in} the language itself, not one in a library.
You will be modifying it later, so make sure you understand what it is doing.
Alternatively, look up the algorithm and implement it yourself. Make sure to
test it on example vectors.

It will probably prove useful to you to separate out the padding and
preprocessing from the actual compression into different functions.

Write a function to authenticate a message under a secret key by using a
secret-prefix MAC, which is simply:
\begin{lstlisting}[language={}]
	SHA1(key || message)
\end{lstlisting}

It may be useful to create a function to verify a given MAC is valid under a
secret key.

Verify that the MAC works; that you cannot modify the message without
invalidating the MAC you've produced, and that you cannot produce a new MAC
without knowing the secret key.

\subsection{Part 2: Break SHA-1 secret-prefix MAC}

Secret-prefix SHA-1 MACs are trivially breakable.

The length extension attack on secret-prefix SHA1 relies on the fact that you
can take the output of SHA-1 and use it as a new starting point for SHA-1, thus
taking an arbitrary SHA-1 hash and ``feeding it more data''.

Since the key is prepended to the data in secret-prefix, any additional data you
feed the SHA-1 hash in this fashion will appear to have been hashed with the
secret key.

To carry out the attack, you'll need to account for the fact that SHA-1 is
``padded'' with the bit-length of the message; your forged message will need to
include that padding. We call this ``glue padding''. The final message you actually forge will be:
\begin{lstlisting}[language={}]
	SHA1(key || original-message || glue-padding || new-message)
\end{lstlisting}

Note that to generate the glue padding, you'll need to know the original bit
length of the message; the message itself is known to the attacker, but the
secret key isn't, so you'll need to guess at it.

You should first write a function that computes the MD padding of a message
passed into it. Verify that it matches the padding generated by your SHA-1
function.

Now, take the SHA-1 secret-prefix MAC of the message you want to forge -- this
is just a SHA-1 hash -- and break it into 32-bit SHA-1 registers.

Modify your hash implementation to allow callers to pass in arbitrary starting
points for ``a'', ``b'', etc. They normally start at magic numbers. With these
registers fixed, hash the additional data you want to forge.

Using this attack, generate a secret-prefix MAC under a secret key (choose a
random word or two from /usr/share/dict/words) of the string:
\begin{lstlisting}
	account=sujeet-shenoi&forward[]=sujeet@utulsa.edu
\end{lstlisting}
Perhaps this is a system that allows users to submit forwarding addresses, in
the clear but authenticated with their passwords (or some other shared secret),
and you have the ability to capture and replay requests\footnote{Remember:
\textit{na\"ive Javascript programmer}.}.

Forge a variant of this message that ends with "\&forward[]=<your email
address>".

\ifinstructornotes
\textbf{Instructor Note:} This comes from challenges 28 and 29 of the Matasano
crypto challenges.
\fi

\section{Rules}

Please do not exploit any non-cryptographic vulnerabilities you may discover on
any machines\footnote{I dunno, maybe you're poking around in your spare time.
You won't find anything useful on any server for this lab.}. You should not need
to attempt to gain any level of access to the machine beyond a TCP connection,
nor should you attempt to mount a Denial of Service attack on any host on the
network.

\section{Grading}
Labs are due Thursday, October 28 at 11:59pm. You will need to turn in
\textbf{all relevant code}, and \textbf{an explanation of how you would modify
the system to prevent this attack}.

\end{document}
