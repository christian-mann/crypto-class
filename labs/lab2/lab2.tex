\newif\ifinstructornotes
\instructornotestrue % comment out to hide instructor notes

\documentclass[11pt,oneside]{article}
\usepackage{geometry}
\usepackage[T1]{fontenc}
\usepackage{lastpage}
\usepackage{fancyhdr}
\usepackage{hyperref}
\usepackage{graphicx}
\usepackage{listings} % to display code
\usepackage{parskip}
\pagestyle{fancy}
%\geometry{letterpaper,tmargin=.75in,bmargin=1.25in,lmargin=.75in,rmargin=.75in,headheight=13.6pt,headsep=0in,footskip=.3in}
\geometry{letterpaper,tmargin=.5in,bmargin=.5in,lmargin=.5in,rmargin=.5in,headheight=13.6pt,headsep=0in,footskip=.3in}

%%%%%%%%%%%%%%%%%%%%%%%%%%%%%%%%%%%%%%%%%%%%%%%%%%%%%%%%%%%%%%%%%%%%%%%%%%%%%%%%

\setlength{\parindent}{0in}
%\setlength{\parskip}{0.0in}
%\setlength{\itemsep}{0in}
%\setlength{\topsep}{0in}
%\setlength{\tabcolsep}{0in}

%%%%%%%%%%%%%%%%%%%%%%%%%%%%%%%%%%%%%%%%%%%%%%%%%%%%%%%%%%%%%%%%%%%%%%%%%%%%%%%%

\newcommand{\class}{Introduction to Cryptography}
\newcommand{\project}{Lab 2 Assignment:\\Block Ciphers and Chosen-Plaintext
Attacks}

\renewcommand{\headrulewidth}{0pt}
%\fancyfoot{}
%\cfoot{\quad \quad \quad \quad \quad \quad \thepage\ of \pageref{LastPage}}
% quads to fix center...
\cfoot{}

%%%%%%%%%%%%%%%%%%%%%%%%%%%%%%%%%%%%%%%%%%%%%%%%%%%%%%%%%%%%%%%%%%%%%%%%%%%%%%%%

% Style
\newcommand{\sectionfont}{phv} % Helvetica
\newcommand{\bodyfont}{ppl} % Palatino

\renewcommand{\section}[1] {
    \vspace{12pt}{\quad\fontfamily{\sectionfont}\selectfont\Large\scshape\textbf{#1}}\\[-10pt]
    \vspace{8pt}\rule{\textwidth}{1pt}\\[-16pt]

    % this space is needed
}

\renewcommand{\subsection}[1] {
    \vspace{12pt}{\fontfamily{\sectionfont}\selectfont\large\scshape\textbf{#1}}\\[-10pt]
    %\vspace{8pt}\rule{\textwidth}{1pt}\\[-16pt]

    % this space is needed
}


% \newcommand{\project}[2] {
%                 \begin{tabular}{p{.25\linewidth}p{.05\linewidth}p{.7\linewidth}}
%                         \textbf{#1} & & #2 \\\\
%                 \end{tabular}
% 
%         % this space is needed
% }

%%%%%%%%%%%%%%%%%%%%%%%%%%%%%%%%%%%%%%%%%%%%%%%%%%%%%%%%%%%%%%%%%%%%%%%%%%%%%%%%

\lstset{
language=python,                % choose the language of the code
%basicstyle=\footnotesize,       % the size of the fonts that are used for the code
%numbers=left,                   % where to put the line-numbers
%numberstyle=\footnotesize,      % the size of the fonts that are used for the line-numbers
%stepnumber=2,                   % the step between two line-numbers. If it's 1 each line will be numbered
%%umbersep=5pt,                  % how far the line-numbers are from the code
%backgroundcolor=\color{white},  % choose the background color. You must add \usepackage{color}
showspaces=false,               % show spaces adding particular underscores
showstringspaces=false,         % underline spaces within strings
showtabs=false,                 % show tabs within strings adding particular underscores
%%frame=single,	                % adds a frame around the code
tabsize=4,	                % sets default tabsize to 2 spaces
%%captionpos=b,                   % sets the caption-position to bottom
%%breaklines=true,                % sets automatic line breaking
%%breakatwhitespace=false,        % sets if automatic breaks should only happen at whitespace
%%title=\lstname,                 % show the filename of files included with \lstinputlisting; also try caption instead of title
%escapeinside={\%*}{*)}          % if you want to add a comment within your code
%morekeywords={*,...}            % if you want to add more keywords to the set
}


\begin{document}

\fontfamily{\bodyfont} \selectfont \small
\thispagestyle{empty}
\begin{center}
    \fontfamily{\sectionfont}\selectfont\huge\scshape\textbf{\class}
\end{center}
\begin{center}
    \fontfamily{\sectionfont}\selectfont\large\scshape\textbf{\project}
\end{center}

\section{Overview}

This lab will involve various attacks on block ciphers. While the cipher used
will be AES, this should not affect your decryption code in any way. Successful
attacks rarely target the core cipher; rather, you will be attacking the
\textit{mode} of encryption used. Particularly, for this project you will be
mounting attacks on the ECB mode of encryption.

Each of these challenges will involve a service on the network, giving out
ciphertexts and returning responses. Your code will need to interact with this
service. In every case, the end goal will be the retrieval of a flag. This will
be accomplished in different ways, depending on the challenge.

Learning to program using sockets is not the goal of this class; you can take Network
Security or CNO for that. Thus, I will provide example socket code for any
languages I know are being used. If you have a language that does not support
network activity, I will be excited to work with you to arrive at a solution.

\section{Assignment}

\subsection{Part 1: ECB Detection}

You will write a function to detect which mode an encryption oracle is
using.

The oracle for this project will prompt for input. When it receives a chunk of
input, it will pad that input on the left and right by 5-10 randomly-chosen
bytes. It will then encrypt the result with a randomly chosen key and IV. The
oracle will use ECB one half of the time and CBC the other half of the time.
This result will be sent back across the network.

At this point, your code should send back ``\texttt{ECB}'' or ``\texttt{CBC}'',
based on your guess.

You will need to guess correctly 100 times in a row, and then the server will
send the flag. On the second incorrect response, the connection will terminate.

\ifinstructornotes
\textbf{Instructor Note:} This is challenge 11 of the Matasano crypto
challenges.
\fi

\subsection{Part 2: ECB Forgery}

Encrypting a ciphertext under a secret key does not provide any level of
integrity verification on its own. You will attack the ciphertext integrity of a
system that uses ECB.

The cryptosystem is a web service\footnote{kludgy raw TCP service also available
if parsing HTTP is too complex}, that uses HTTP cookies. Cookie data is kept on
the client, and kept encrypted. You should modify this encrypted data to change
your role from \texttt{user} to \texttt{admin}.

%The system uses a single AES key, and encrypts in ECB mode. Hint for the
%challenge -- \textit{copy-and-paste}.

\ifinstructornotes
\textbf{Instructor Note:} This is challenge 13 of the Matasano crypto
challenges.
\fi

\subsection{Part 3: ECB Decryption}

You will use your knowledge of ECB to attack a cryptosystem by decrypting secret
plaintext. This process roughly models various attacks on protocols such as HTTP
and SSH.

The oracle for this project will compute the following function:

\begin{lstlisting}
    AES-256-ECB(your-string || secret-data || padding, secret-key)
\end{lstlisting}

You can decrypt \texttt{secret-data} with repeated calls to the oracle. Here's
how:

\begin{enumerate}
	\item Feed progressively longer strings to the oracle (start with ``A'',
		then ``AA'', and so on). Discover the block size of the oracle. You
		already know that AES has a 16-byte block, but do it anyway.
	
	\item Detect that the oracle is performing ECB. Again, you already know
		this, but do it anyway.
	
	\item Craft an input that is exactly one byte short of a full block. Think
		about what the oracle will put in that last byte.

	\item Make a mapping of every possible last byte by feeding different
		strings to the oracle; for instance, ``AAAAAAAA'', ``AAAAAAAB'',
		``AAAAAAAC'', remembering the first block returned from every
		invocation.
	
	\item Match the output of the one-byte-short input to one of the entries in
		your dictionary. You have discovered the first byte of unknown-string.
	
	\item Repeat for the next byte.
\end{enumerate}

\ifinstructornotes
\textbf{Instructor Note:} This is challenge 12 of the Matasano crypto
challenges.
\fi

\section{Rules}

Please do not exploit any non-cryptographic vulnerabilities you may discover on
any machines. You should not need to attempt\footnote{I hate telling people they
are not allowed to hack my services. If you want to try, please go ahead. If you
find something, just tell me, and don't use it to cheat on the lab.} to gain any
level of access to the machine beyond a TCP connection, nor should you attempt
to mount a Denial of Service attack on any host on the network.

\section{Grading}
Labs are due Thursday, September 10 at 11:59pm. You will need to turn in all
recovered keys.

\end{document}
