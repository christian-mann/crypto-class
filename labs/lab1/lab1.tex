\documentclass[11pt,oneside]{article}
\usepackage{geometry}
\usepackage[T1]{fontenc}
\usepackage{lastpage}
\usepackage{fancyhdr}
\usepackage{hyperref}
\usepackage{graphicx}
\usepackage{listings} % to display code
\pagestyle{fancy}
%\geometry{letterpaper,tmargin=.75in,bmargin=1.25in,lmargin=.75in,rmargin=.75in,headheight=13.6pt,headsep=0in,footskip=.3in}
\geometry{letterpaper,tmargin=.5in,bmargin=.5in,lmargin=.5in,rmargin=.5in,headheight=13.6pt,headsep=0in,footskip=.3in}

%%%%%%%%%%%%%%%%%%%%%%%%%%%%%%%%%%%%%%%%%%%%%%%%%%%%%%%%%%%%%%%%%%%%%%%%%%%%%%%%

\setlength{\parindent}{0in}
\setlength{\parskip}{0.0in}
%\setlength{\itemsep}{0in}
%\setlength{\topsep}{0in}
%\setlength{\tabcolsep}{0in}

%%%%%%%%%%%%%%%%%%%%%%%%%%%%%%%%%%%%%%%%%%%%%%%%%%%%%%%%%%%%%%%%%%%%%%%%%%%%%%%%

\newcommand{\class}{Introduction to Cryptography}
\newcommand{\project}{Lab 1 Assignment:\\Ancient Cryptography and
Ciphertext-Only Attacks}

\renewcommand{\headrulewidth}{0pt}
%\fancyfoot{}
%\cfoot{\quad \quad \quad \quad \quad \quad \thepage\ of \pageref{LastPage}}
% quads to fix center...
\cfoot{}

%%%%%%%%%%%%%%%%%%%%%%%%%%%%%%%%%%%%%%%%%%%%%%%%%%%%%%%%%%%%%%%%%%%%%%%%%%%%%%%%

% Style
\newcommand{\sectionfont}{phv} % Helvetica
\newcommand{\bodyfont}{ppl} % Palatino

\renewcommand{\section}[1] {
    \vspace{12pt}{\quad\fontfamily{\sectionfont}\selectfont\Large\scshape\textbf{#1}}\\[-10pt]
    \vspace{8pt}\rule{\textwidth}{1pt}\\[-16pt]

    % this space is needed
}

\renewcommand{\subsection}[1] {
    \vspace{12pt}{\fontfamily{\sectionfont}\selectfont\large\scshape\textbf{#1}}\\[-10pt]
    %\vspace{8pt}\rule{\textwidth}{1pt}\\[-16pt]

    % this space is needed
}


% \newcommand{\project}[2] {
%                 \begin{tabular}{p{.25\linewidth}p{.05\linewidth}p{.7\linewidth}}
%                         \textbf{#1} & & #2 \\\\
%                 \end{tabular}
% 
%         % this space is needed
% }

%%%%%%%%%%%%%%%%%%%%%%%%%%%%%%%%%%%%%%%%%%%%%%%%%%%%%%%%%%%%%%%%%%%%%%%%%%%%%%%%

\lstset{
language=python,                % choose the language of the code
%basicstyle=\footnotesize,       % the size of the fonts that are used for the code
%numbers=left,                   % where to put the line-numbers
%numberstyle=\footnotesize,      % the size of the fonts that are used for the line-numbers
%stepnumber=2,                   % the step between two line-numbers. If it's 1 each line will be numbered
%%umbersep=5pt,                  % how far the line-numbers are from the code
%backgroundcolor=\color{white},  % choose the background color. You must add \usepackage{color}
showspaces=false,               % show spaces adding particular underscores
showstringspaces=false,         % underline spaces within strings
showtabs=false,                 % show tabs within strings adding particular underscores
%%frame=single,	                % adds a frame around the code
tabsize=4,	                % sets default tabsize to 2 spaces
%%captionpos=b,                   % sets the caption-position to bottom
%%breaklines=true,                % sets automatic line breaking
%%breakatwhitespace=false,        % sets if automatic breaks should only happen at whitespace
%%title=\lstname,                 % show the filename of files included with \lstinputlisting; also try caption instead of title
%escapeinside={\%*}{*)}          % if you want to add a comment within your code
%morekeywords={*,...}            % if you want to add more keywords to the set
}


\begin{document}

\fontfamily{\bodyfont} \selectfont \small
\thispagestyle{empty}
\begin{center}
    \fontfamily{\sectionfont}\selectfont\huge\scshape\textbf{\class}
\end{center}
\begin{center}
    \fontfamily{\sectionfont}\selectfont\large\scshape\textbf{\project}
\end{center}

\section{Overview}

The goal of this lab is to be an exercise in ciphertext-only cryptanalysis.
You will be given a sequence of ciphertexts with known encryption methods, but
unknown keys.

You will be required to crack the encryption on each file. The decrypted
plaintext alone will not be sufficient; you will be required to submit your code
as well, or an explanation of how you cracked it, if you used pencil-and-paper
methods.

\section{Assignment}

I will distribute a zip file containing each of the ciphertexts. They are
encrypted as follows:

\subsection{Part 1}
This file has been encrypted with a Caesar cipher.

\subsection{Part 2}
This file has been encrypted with a monoalphabetic cipher. Whitespace can be
trusted, but you should not need to consider it in your analysis.

You will need some way of scoring a candidate English plaintext. Character
frequency is a good metric. Evaluate each output and choose the one with the
best score.

\subsection{Part 3}
This file has been encrypted with a single-byte XOR key, and then
base64-encoded. You will need to base64-decode the file before automated
analysis. This is very similar to part 2, but more closely resembles modern
cryptography.

\subsection{Part 4}
This file has been encrypted with a multiple-byte XOR key, in a similar way to
Viginere -- the key is repeated for the length of the file. Again, this file has
been base64-encoded before distribution. Below is a suggested decryption
strategy.

\begin{enumerate}
	\item Consider KEYSIZE between 2 and some upper bound, say 50.
	\item Write a function to compute the edit (Hamming) distance between two strings --
		the number of different bits.
	\item For each value of KEYSIZE, take the first KEYSIZE bytes, and the
		second KEYSIZE bytes, and determine the edit distance between them.
	\item Because English text, as encoded in ASCII, is very self-similar, the
		KEYSIZE with the smallest edit distance is probably the right size.
	\item Break the text into blocks of KEYSIZE length.
	\item Transpose the blocks. You now have KEYSIZE slices.
	\item Each slice is now a single-byte XOR cipher. You already have code to
		solve this.
	\item Put the key bytes together and you have the key.
\end{enumerate}

\subsection{Part 5 - Bonus}
This file has been encrypted with a Viginere cipher. I will not outline a
strategy for this; there are plenty available online.

\section{Suggested Strategy}
The parts are arranged roughly in order of difficulty. Before beginning to crack
any of the encryption techniques, I recommend you begin by writing an
\textit{encryptor} for each of them, so that you can generate ciphertext for
known plaintext, and a decryptor, so that you can decrypt a file, given a known
key.

\section{Grading}
Labs are due September 3 at 11:59pm. Labs will be graded on: proper decryption
of ciphertext, and validity of technique applied. Code quality may factor
into the grade.

\end{document}
