\newif\ifinstructornotes
\instructornotestrue % comment out to hide instructor notes

\documentclass[11pt,oneside]{article}
\usepackage{geometry}
\usepackage[T1]{fontenc}
\usepackage{lastpage}
\usepackage{fancyhdr}
\usepackage{hyperref}
\usepackage{graphicx}
\usepackage{listings} % to display code
\usepackage{parskip}
\usepackage{url}
\pagestyle{fancy}
%\geometry{letterpaper,tmargin=.75in,bmargin=1.25in,lmargin=.75in,rmargin=.75in,headheight=13.6pt,headsep=0in,footskip=.3in}
\geometry{letterpaper,tmargin=.5in,bmargin=.5in,lmargin=.5in,rmargin=.5in,headheight=13.6pt,headsep=0in,footskip=.3in}

%%%%%%%%%%%%%%%%%%%%%%%%%%%%%%%%%%%%%%%%%%%%%%%%%%%%%%%%%%%%%%%%%%%%%%%%%%%%%%%%

\setlength{\parindent}{0in}
%\setlength{\parskip}{0.0in}
%\setlength{\itemsep}{0in}
%\setlength{\topsep}{0in}
%\setlength{\tabcolsep}{0in}

%%%%%%%%%%%%%%%%%%%%%%%%%%%%%%%%%%%%%%%%%%%%%%%%%%%%%%%%%%%%%%%%%%%%%%%%%%%%%%%%

\newcommand{\class}{Introduction to Cryptography}
\newcommand{\project}{Lab 7 Assignment:\\Random Number Generators}

\renewcommand{\headrulewidth}{0pt}
%\fancyfoot{}
%\cfoot{\quad \quad \quad \quad \quad \quad \thepage\ of \pageref{LastPage}}
% quads to fix center...
\cfoot{}

%%%%%%%%%%%%%%%%%%%%%%%%%%%%%%%%%%%%%%%%%%%%%%%%%%%%%%%%%%%%%%%%%%%%%%%%%%%%%%%%

% Style
\newcommand{\sectionfont}{phv} % Helvetica
\newcommand{\bodyfont}{ppl} % Palatino

\renewcommand{\section}[1] {
    \vspace{12pt}{\quad\fontfamily{\sectionfont}\selectfont\Large\scshape\textbf{#1}}\\[-10pt]
    \vspace{8pt}\rule{\textwidth}{1pt}\\[-16pt]

    % this space is needed
}

\renewcommand{\subsection}[1] {
    \vspace{12pt}{\fontfamily{\sectionfont}\selectfont\large\scshape\textbf{#1}}\\[-10pt]
    %\vspace{8pt}\rule{\textwidth}{1pt}\\[-16pt]

    % this space is needed
}


% \newcommand{\project}[2] {
%                 \begin{tabular}{p{.25\linewidth}p{.05\linewidth}p{.7\linewidth}}
%                         \textbf{#1} & & #2 \\\\
%                 \end{tabular}
% 
%         % this space is needed
% }

%%%%%%%%%%%%%%%%%%%%%%%%%%%%%%%%%%%%%%%%%%%%%%%%%%%%%%%%%%%%%%%%%%%%%%%%%%%%%%%%

\lstset{
language=python,                % choose the language of the code
%basicstyle=\footnotesize,       % the size of the fonts that are used for the code
%numbers=left,                   % where to put the line-numbers
%numberstyle=\footnotesize,      % the size of the fonts that are used for the line-numbers
%stepnumber=2,                   % the step between two line-numbers. If it's 1 each line will be numbered
%%umbersep=5pt,                  % how far the line-numbers are from the code
%backgroundcolor=\color{white},  % choose the background color. You must add \usepackage{color}
showspaces=false,               % show spaces adding particular underscores
showstringspaces=false,         % underline spaces within strings
showtabs=false,                 % show tabs within strings adding particular underscores
%%frame=single,	                % adds a frame around the code
tabsize=4,	                % sets default tabsize to 2 spaces
%%captionpos=b,                   % sets the caption-position to bottom
%%breaklines=true,                % sets automatic line breaking
%%breakatwhitespace=false,        % sets if automatic breaks should only happen at whitespace
%%title=\lstname,                 % show the filename of files included with \lstinputlisting; also try caption instead of title
%escapeinside={\%*}{*)}          % if you want to add a comment within your code
%morekeywords={*,...}            % if you want to add more keywords to the set
}


\begin{document}

\fontfamily{\bodyfont} \selectfont \small
\thispagestyle{empty}
\begin{center}
    \fontfamily{\sectionfont}\selectfont\huge\scshape\textbf{\class}
\end{center}
\begin{center}
    \fontfamily{\sectionfont}\selectfont\large\scshape\textbf{\project}
\end{center}

\section{Overview}

In this lab, you will directly attack a random number generator. While the
particulars of this lab will only deal with one random number generator, the
techniques and concepts should be applicable to a wide range of RNGs.

The Mersenne Twister is a random number generator that is designed to have a
very long period, be relatively fast to compute, and have a high degree of
randomness. However, it is not suitable for cryptographic applications, as it
can be very predictable, given enough information.

In the first section of this lab, you will recover the seed that was used for
the generator, given a rough idea of what the seed value might be, and a handful
of outputs.

In the second section, you will recover the internal state of the generator,
given a sequence of consecutive outputs.

\section{Assignment}

Just as in every other lab, you will be given a library that contains various
utility functions, as well as an implementation of 32-bit MT19937. I would
encourage you to use this by itself while writing your scripts; the extra
infrastructure surrounding it can be cumbersome.

\subsection{Part 1: Seed Recovery}

The oracle for this lab does the following:
\begin{itemize}
	\item Waits a random number of seconds between 40 and 1000.
	\item Seeds the RNG with the current UNIX timestamp, as seen by the server.
	\item Waits a random number of seconds between 40 and 500 (again).
	\item Returns a proxy to that RNG. This will allow you to generate as many
		numbers as you wish, but you should not need more than a handful.
\end{itemize}

If you can correctly predict the seed that was used, you will earn a flag.
Again, example code will be given.

If you do the math, you will notice that some of these wait times are very long.
You may wish to practice with a shorter timeout, or by cheating locally. The
server time should be very close to the correct time\footnote{The server uses
timestamps in UTC.}.

\ifinstructornotes
\textbf{Instructor Note:} This is challenge 23 of the Matasano crypto
challenges.
\fi

\subsection{Part 2: State Reconstruction}

The internal state of MT19937 consists of 624 32-bit integers.

For each batch of 624 outputs, MT permutes that internal state. By permuting
that state regularly, MT19937 achieves a period of $2^{19937}$, which is Big.

Each time MT19937 is tapped, an element of its internal state is subject to a
\textit{temper}ing function that diffuses bits through the result. The tempering
function is invertible; you can write an \textit{untemper} function that takes
an MT19937 output and transforms it back into the corresponding element of the
MT19937 state array.

\ifinstructornotes
\textbf{Instructor Note:} To invert the temper function, apply the inverse of
each of the operations in the temper function in reverse order. There are two
kinds of operations in the temper transform, each applied twice: one is an XOR
against a right-shifted value, and the other is an XOR against a left-shifted
value ANDed with a magic number. So you'll need code to invert the ``right'' and
the ``left'' operation.
\fi

Once you have \textit{untemper} working, obtain an MT19937 generator from the
oracle, tap it for 624 outputs, untemper each of them to recreate the state of
the generator, and splice that into a new instance of the MT19937 generator.

The new spliced generator should predict the values of the original. You will
need to predict five consecutive values in order to earn a flag.

\ifinstructornotes
\textbf{Instructor Note:} This is challenge 23 of the Matasano crypto
challenges.
\fi

\section{Rules}

Standard rules apply. Please do not exploit any non-cryptographic
vulnerabilities you may discover on any machines. You should not need to attempt
to gain any level of access to the machine beyond a TCP connection, nor should
you attempt to mount a Denial of Service attack on any host on the network.

\section{Grading}
Labs are due Thursday, November 12 at 11:59pm. You will need to turn in
all relevant code, and the recovered flags.

\end{document}
